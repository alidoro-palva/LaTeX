\documentclass[a5paper,10pt]{article}
\oddsidemargin=0pt
\hoffset=-1.5cm
\voffset=-1.5cm
\topmargin=-1.5cm
\textwidth=12.8cm
\textheight=18.6cm
\usepackage[utf8]{inputenc}
\usepackage[T2A]{fontenc}
\usepackage[russian]{babel}
\usepackage[T2A]{fontenc}
\usepackage{latexsym}
\usepackage{amssymb}
\usepackage{amsmath}
\usepackage{bm}
\usepackage{graphicx}

\begin{document}

\noindent {\it 
Ким Г.Д., Крицков Л.В. -- Алгебра и аналитическая геометрия. Теоремы и задачи. Том I (2007)}

\bigskip
\noindent
Вычислить определители, приводя их матрицы к треугольному виду.

\medskip
\noindent
{\bf 7.30.}
$\begin{vmatrix}
1&n&n&\ldots&n&n\\
n&2&n&\ldots&n&n\\
n&n&3&\ldots&n&n\\
.&.&.&.&.&.\\
n&n&n&\ldots&n-1&n\\
n&n&n&\ldots&n&n\\
\end{vmatrix}$.

\medskip
\noindent
$\blacktriangleleft$ Вычтем последнюю строку из всех остальных. Получим\\[3pt]
$\begin{vmatrix}
1-n&0&0&\ldots&0&0\\
0&2-n&0&\ldots&0&0\\
0&0&3-n&\ldots&0&0\\
.&.&.&.&.&.\\
0&0&0&\ldots&-1&0\\
n&n&n&\ldots&n&n\\
\end{vmatrix}=(-1)^{n-1}n!$
$\blacktriangleright$

\bigskip
\noindent
Исследовать на совместность и найти общее решение системы уравнений.

\medskip
\noindent
{\bf 21.6}
$\begin{cases}
2x_1-x_2+3x_3&=3\\
3x_1+x_2-5x_3&=0\\
4x_1-x_2+x_3&=3\\
x_1+3x_2-13x_3&=-6
\end{cases}$.

\medskip
\noindent
$\blacktriangleleft$ Приводим расширенную матрицу системы к ступенчатому виду. Меняем местами первую и четвертую строки, затем вычитаем из второй строки первую, умноженную на $3$, из третьей первую, умноженную на $4$ из четвертой первую, умноженную на $2$.\\[3pt]
$\left(\begin{array}{rrr|r}
2& -1& 3& 3\\
3& 1& -5& 0\\
4& -1& 1& 3\\
1& 3& -13& -6
\end{array}\right)$
$\sim$
$\left(\begin{array}{rrr|r}
1& 3& -13& -6\\
3& 1& -5& 0\\
4& -1& 1& 3\\
2& -1& 3& 3
\end{array}\right)$
$\sim$
$\left(\begin{array}{rrr|r}
1& 3& -13& -6\\
0& -8& 34& 18\\
0& -13& 53& 27\\
0& -7& 29& 15
\end{array}\right)$
$\sim$\\[3pt]
Далее вычитаем из второй строки четвертую и из третьей строки удвоенную четвертую, затем прибавляем к третьей строке вторую, а из четвертой строки вычитаем вторую, умноженную на $7$, затем делим на -6 четвертую строку, меняем ее местами с третьей и меняем знак у второй строки.\\[3pt]
$\left(\begin{array}{rrr|r}
1& 3& -13& -6\\
0& -1& 5& 3\\
0& 1& -5& -3\\
0& -7& 29& 15
\end{array}\right)$
$\sim$
$\left(\begin{array}{rrr|r}
1& 3& -13& -6\\
0& -1& 5& 3\\
0& 0& 0& 0\\
0& 0& -6& -6
\end{array}\right)$
$\sim$
$\left(\begin{array}{rrr|r}
1& 3& -13& -6\\
0& 1& -5& -3\\
0& 0& 1& 1\\
0& 0& 0& 0
\end{array}\right)$
$\sim$\\[3pt]
Далее прибавляем к второй строке третью, умноженную на $5$, и к первой строке третью, умноженную на $13$, затем от первой строки отнимаем вторую, умноженную на $3$.\\[3pt]
$\left(\begin{array}{rrr|r}
1& 3& 0& 7\\
0& 1& 0& 2\\
0& 0& 1& 1\\
0& 0& 0& 0
\end{array}\right)$
$\sim$
$\left(\begin{array}{rrr|r}
1& 0& 0& 1\\
0& 1& 0& 2\\
0& 0& 1& 1\\
0& 0& 0& 0
\end{array}\right)$.
\\[3pt]
Ответ: $x_1=1,\ x_2=2,\ x_3=1$.
$\blacktriangleright$

\medskip
\noindent
{\bf 21.7}
$\begin{cases}
3x_1+x_2-2x_3+x_4-x_5&=1\\
2x_1-x_2+7x_3-3x_4+5x_5&=2\\
x_1+3x_2-2x_3+5x_4-7x_5&=3\\
3x_1-2x_2+7x_3-5x_4+8x_5&=3
\end{cases}$

\medskip
\noindent
$\blacktriangleleft$ Приводим расширенную матрицу системы к ступенчатому виду. Сначала мы из четвертой строки вычитаем первую, а затем меняем местами первую и третью строки.
\\[3pt]
$\left(\begin{array}{rrrrr|r}
3& 1& -2& 1& -1& 1\\
2& -1& 7& -3& 5& 2\\
1& 3& -2& 5& -7& 3\\
3& -2& 7& -5& 8& 3
\end{array}\right)$
$\sim$
$\left(\begin{array}{rrrrr|r}
1& 3& -2& 5& -7& 3\\
2& -1& 7& -3& 5& 2\\
3& 1& -2& 1& -1& 1\\
0& -3& 9& -6& 9& 2
\end{array}\right)$
$\sim$\\[3pt]
Далее вычитаем из третьей строки утроенную первую, а из второй строки удвоенную первую. На следующем шаге из второй строки вычтем третью.\\[3pt]
$\left(\begin{array}{rrrrr|r}
1& 3& -2& 5& -7& 3\\
0& -7& 11& -13& 19& -4\\
0& -8& 4& -14& 20& -8\\
0& -3& 9& -6& 9& 2
\end{array}\right)$
$\sim$
$\left(\begin{array}{rrrrr|r}
1& 3& -2& 5& -7& 3\\
0& 1& 7& 1& -1& 4\\
0& -8& 4& -14& 20& -8\\
0& -3& 9& -6& 9& 2
\end{array}\right)$
$\sim$\\[3pt]
Прибавляем к третьей строке вторую, умноженную на 8, и к четвертой вторую строку, умноженную на 3. Далее из третьей строки вычитаем удвоенную вторую.\\[3pt]
$\left(\begin{array}{rrrrr|r}
1& 3& -2& 5& -7& 3\\
0& 1& 7& 1& -1& 4\\
0& 0& 60& -6& 12& 24\\
0& 0& 30& -3& 6& 14
\end{array}\right)$
$\sim$
$\left(\begin{array}{rrrrr|r}
1& 3& -2& 5& -7& 3\\
0& 1& 7& 1& -1& 4\\
0& 0& 0& 0& 0& -4\\
0& 0& 30& -3& 6& 14
\end{array}\right)$.
\\[3pt]
Теперь третья строка соответствует несовместному линейному уравнению.\\
Ответ: Система несовместна.
$\blacktriangleright$

\medskip
\noindent
{\bf 21.15}
$\begin{cases}
x_1+x_2+x_3+x_4+x_5&=7\\
3x_1+2x_2+x_3+x_4-3x_5&=-2\\
x_2+2x_3+2x_4+6x_5&=23\\
5x_1+4x_2+3x_3+3x_4-x_5&=12
\end{cases}$

\medskip
\noindent
$\blacktriangleleft$ Приводим расширенную матрицу системы к ступенчатому виду. Из второй строки вычитаем утроенную первую, из четвертой строки вычитаем первую, умноженную на $5$.\\[3pt]
$\left(\begin{array}{rrrrr|r}
1& 1& 1& 1& 1& 7\\
3& 2& 1& 1& -3& -2\\
0& 1& 2& 2& 6& 23\\
5& 4& 3& 3& -1& 12
\end{array}\right)$
$\sim$
$\left(\begin{array}{rrrrr|r}
1& 1& 1& 1& 1& 7\\
0& -1& -2& -2& -6& -23\\
0& 1& 2& 2& 6& 23\\
0& -1& -2& -2& -6& -23
\end{array}\right)$
$\sim$\\[3pt]
Ко второй и четвертой строке прибавляем третью. Затем меняем местами вторую и третью строки.\\[3pt]
$\left(\begin{array}{rrrrr|r}
1& 1& 1& 1& 1& 7\\
0& 0& 0& 0& 0& 0\\
0& 1& 2& 2& 6& 23\\
0& 0& 0& 0& 0& 0
\end{array}\right)$
$\sim$
$\left(\begin{array}{rrrrr|r}
1& 1& 1& 1& 1& 7\\
0& 1& 2& 2& 6& 23\\
0& 0& 0& 0& 0& 0\\
0& 0& 0& 0& 0& 0
\end{array}\right)$
$\sim$\\[3pt]
Вычитаем из первой строки вторую.\\[3pt]
$\left(\begin{array}{rrrrr|r}
1& 0& -1& -1& -5& -16\\
0& 1& 2& 2& 6& 23\\
0& 0& 0& 0& 0& 0\\
0& 0& 0& 0& 0& 0
\end{array}\right)$. Теперь можем написать ответ.\\[3pt]
Ответ: $x_1=-16+x_3+x_4+5x_5,\ x_2=23-2x_3-2x_4-6x_5,\ x_3,x_4,x_5\in \mathbb R$.
$\blacktriangleright$

\bigskip
\noindent
Найти общее решение следующих систем уравнений через их фундаментальные системы решений.

\medskip
\noindent
{\bf 22.16.}
$\begin{cases}
3x_1+5x_2+3x_3+2x_4+x_5&=0\\
5x_1+7x_2+6x_3+4x_4+3x_5&=0\\
7x_1+9x_2+9x_3+6x_4+5x_5&=0\\
4x_1+8x_2+3x_3+2x_4&=0
\end{cases}$

\medskip
\noindent
$\blacktriangleleft$ Приводим расширенную матрицу системы к ступенчатому виду. Вычтем четвертую строку из первой.\\[3pt]
$\left(\begin{array}{rrrrr|r}
3& 5& 3& 2& 1& 0\\
5& 7& 6& 4& 3& 0\\
7& 9& 9& 6& 5& 0\\
4& 8& 3& 2& 0& 0
\end{array}\right)$
$\sim$
$\left(\begin{array}{rrrrr|r}
-1& -3& 0& 0& 1& 0\\
5& 7& 6& 4& 3& 0\\
7& 9& 9& 6& 5& 0\\
4& 8& 3& 2& 0& 0
\end{array}\right)$
$\sim$\\[3pt]
 Прибавим ко второй строке первую, умноженную на $5$, к третьей первую, умноженную на $7$, к четвертой первую, умноженную на $4$, затем сменим знак у первой строки, разделим вторую строку на $-2$, а третью строку на $-3$.\\[3pt]
$\left(\begin{array}{rrrrr|r}
-1& -3& 0& 0& 1& 0\\
0& -8& 6& 4& 8& 0\\
0& -12& 9& 6& 12& 0\\
0& -4& 3& 2& 4& 0
\end{array}\right)$
$\sim$
$\left(\begin{array}{rrrrr|r}
1& 3& 0& 0& -1& 0\\
0& 4& -3& -2& -4& 0\\
0& 4& -3& -2& -4& 0\\
0& 4& -3& -2& -4& 0
\end{array}\right)$
$\sim$\\[3pt]
Из третьей и четвертой строки вычтем вторую, затем вторую строку разделим на $4$ и опустим нулевые строки.\\[3pt]
$\left(\begin{array}{rrrrr|r}
1& 3& 0& 0& -1& 0\\
0& 4& -3& -2& -4& 0\\
0& 0& 0& 0& 0& 0\\
0& 0& 0& 0& 0& 0
\end{array}\right)$
$\sim$
$\left(\begin{array}{rrrrr|r}
1& 3& 0& 0& -1& 0\\
0& 1& -\frac34& -\frac12& -1& 0
\end{array}\right)$
$\sim$\\[3pt]
Из первой строки вычтем утроенную вторую.\\[3pt]
$\left(\begin{array}{rrrrr|r}
1& 0& \frac94& \frac32& 2& 0\\[3pt]
0& 1& -\frac34& -\frac12& -1& 0
\end{array}\right)$.\\[3pt]
Для свободных переменных $x_3,x_4,x_5$ выберем значения, соответствующие стандартному базису  $(1,0,0),(0,1,0),(0,0,1)$, и найдем соответствующие значения переменных $x_1,x_2$. Получаем следующие базисные решения.\\
$(x_1,x_2,x_3,x_4,x_5)=(-\frac94,\frac34,1,0,0)\parallel(-9,3,4,0,0)$,\\[3pt]
$(x_1,x_2,x_3,x_4,x_5)=(-\frac32,\frac12,0,1,0)\parallel(-3,1,0,2,0)$,\\[3pt]
$(x_1,x_2,x_3,x_4,x_5)=(-2,1,0,0,1)$\\
В качестве ответа берем их произвольную линейную комбинацию.\\
Ответ: $x=\alpha_1(-9,3,4,0,0)^T+\alpha_2(-3,1,0,2,0)^T+\alpha_3(-2,1,0,0,1)^T$.
$\blacktriangleright$

\bigskip
\noindent
{\bf 39.9.} Доказать, что конечное множество $G$, в котором определена ассоциативная алгебраическая операция, подчиняющаяся закону сокращения слева и справа, является группой.

\medskip
\noindent
$\blacktriangleleft$ Поскольку $G$ конечно, для любого элемента $a$ среди бесконечного множества его степеней $a,\ a^2,\ a^3,\ \ldots$ найдутся две совпадающих. Приравняв их и произведя максимально возможное число сокращений получим $a^{k_a+1}=a,\ k_a>0$. $a^{k_a}$ является единицей группы. В самом деле. Для произвольного элемента $b$ имеем $ba^{k_a}a=ba$, поэтому $ba^{k_a}=b$ и $aa^{k_a}b=ab$, поэтому $a^{k_a}b=b$. Единица единственна и может быть получена аналогичными рассуждениями как степень любого элемента $b$. Пусть, например, $b^{k_b+1}=b$. Тогда $b^{k_b}$ является единицей и элемент $b^{k_b-1}$ является обратным к $b$.
$\blacktriangleright$

\bigskip
\noindent
{\bf 39.10.} Доказать, что если $a^2=1$ для любого элемента $a$ группы $G$, то эта группа абелева.

\medskip
\noindent
$\blacktriangleleft$ Для любых элементов группы $a,b$ имеем $ab=b^2aba^2=b(ba)^2a=ba$.
$\blacktriangleright$

\bigskip
\noindent
{\bf 39.14.} Доказать, что:\\
а) группа $\mathbb R_+$ положительных действительных чисел по умножению изоморфна группе $\mathbb R$ всех действительных чисел по сложению;\\
б) группа $\mathbb Q_+$ положительных рациональных чисел по умножению не изоморфна группе $\mathbb Q$ всех рациональных чисел по сложению.

\medskip
\noindent
$\blacktriangleleft$
а) В качестве изоморфизма возьмем функцию $\varphi(a)=e^a$. Проверяем основное свойство изоморфизма. $\varphi(a+b)=e^{a+b}=e^a\cdot e^b=\varphi(a)\cdot\varphi(b)$. Обратным отображением служит отображение $\varphi^{-1}(b)=\ln(b)$.\\
б) Предположим противное. Пусть $\varphi$ изоморфизм. Тогда существует число $a$, такое что $\varphi(a)=2$. Имеем
$\displaystyle 2=\varphi(a)=\varphi\left(\frac a2+\frac a2\right)=\varphi\left(\frac a2\right)\cdot\varphi\left(\frac a2\right)=\varphi\left(\frac a2\right)^2$. Получили, что квадрат некоторого рационального числа равен двум, что, как мы знаем, невозможно.
$\blacktriangleright$

\bigskip
\noindent
{\bf 40.26.} Привести примеры колец матриц специального вида, обладающих не\-сколькими правыми или несколькими левыми единицами.

\medskip
\noindent
$\blacktriangleleft$ Первый пример это кольцо всевозможных числовых матриц вида $\begin{pmatrix}a&b\\0&0\end{pmatrix}$. В нем любая матрица вида $\begin{pmatrix}1&b\\0&0\end{pmatrix}$ будет левой единицей. Транспонируя ситуацию, получаем кольцо матриц вида $\begin{pmatrix}a&0\\b&0\end{pmatrix}$. В нем правыми единицами будут матрицы вида $\begin{pmatrix}1&0\\b&0\end{pmatrix}$. $\blacktriangleright$

\bigskip
\noindent
{\bf 40.31.} Показать, что поле матриц вида $\begin{pmatrix}a&b\\2b&a\end{pmatrix},\ \ a,b\in \mathbb Q$, изоморфно полю чисел вида $a+b\sqrt2$, где $a,b\in \mathbb Q$.

\medskip
\noindent
$\blacktriangleleft$ Вид изоморфизма $\varphi$ подсказан обозначениями, использованными в задаче.
$\varphi\left(\begin{bmatrix}a&b\\2b&a\end{bmatrix}\right)=a+b\sqrt2$. Нам надо проверить, что это изоморфизм. Пишем:
$\varphi\left(\begin{bmatrix}a&b\\2b&a\end{bmatrix}+\begin{bmatrix}c&d\\2d&c\end{bmatrix}\right)=\varphi\left(\begin{bmatrix}a+c&b+d\\2b+2d&a+c\end{bmatrix}\right)=a+c+(b+d)\sqrt2=\\[3pt]
=a+b\sqrt2+c+d\sqrt2=\varphi\left(\begin{bmatrix}a&b\\2b&a\end{bmatrix}\right)+\varphi\left(\begin{bmatrix}c&d\\2d&c\end{bmatrix}\right)$. То же для умножения.\\
$\varphi\left(\begin{bmatrix}a&b\\2b&a\end{bmatrix}\cdot\begin{bmatrix}c&d\\2d&c\end{bmatrix}\right)=\varphi\left(\begin{bmatrix}ac+2bd&ad+bc\\2(ad+bc)&ac+2bd\end{bmatrix}\right)=ac+2bd+(ad+bc)\sqrt2=\\[3pt]
=ac+\sqrt2\sqrt2bd+ad\sqrt2+bc\sqrt2=a(c+d\sqrt2)+b\sqrt2(c+d\sqrt2)=(a+b\sqrt2)(c+d\sqrt2)=\\[3pt]
\varphi\left(\begin{bmatrix}a&b\\2b&a\end{bmatrix}\right)\cdot\varphi\left(\begin{bmatrix}c&d\\2d&c\end{bmatrix}\right)$. Далее нам надо доказать существование обратного отображения. Пусть имеется число $x$ вида $a+b\sqrt2$. Чтобы иметь право написать $\varphi^{-1}(x)=\left(\begin{bmatrix}a&b\\2b&a\end{bmatrix}\right)$, прежде всего надо доказать, что $a$ и $b$ могут быть однозначно определены по значению $x$. И в самом деле, пусть имеется неоднозначность, то есть $a+b\sqrt2=c+d\sqrt2$. Если $b=d$, то сокращая получаем $a=c$, и неоднозначности не будет. Итак $b\ne d$. Теперь из равенства разных представлений числа $x$ получаем $a-c=(d-b)\sqrt2$ и $\displaystyle \sqrt2=\frac{a-c}{d-b}$, но последнее равенство невозможно, поскольку число $\sqrt2$ иррационально. 
$\blacktriangleright$

\end{document}
