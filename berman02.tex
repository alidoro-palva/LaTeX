\documentclass[a5paper,10pt]{article}
\oddsidemargin=0pt
\hoffset=-1.5cm
\voffset=-1.5cm
\topmargin=-1.5cm
\textwidth=12.8cm
\textheight=18.6cm
\usepackage[utf8]{inputenc}
\usepackage[russian]{babel}
\usepackage[T2A]{fontenc}
\usepackage{fontspec}
\setmainfont{Times New Roman}
\usepackage{latexsym}
\usepackage{amssymb}
\usepackage{amsmath}
\usepackage{bm}
\usepackage{graphicx}

\begin{document}

\noindent {\it Берман. Сборник задач по курсу математического анализа.
Издание двадцатое. М., 1985.}

\bigskip
\section* {Глава II. Понятие о пределе}

\medskip
\noindent Вычислить пределы:

\medskip
\noindent{\bf 286.} $\displaystyle \lim_{x\to\infty}
\left(\frac{x^3}{2x^2-1}-\frac{x^2}{2x+1}\right)=
\lim_{x\to\infty}\frac{2x^4+x^3-2x^4+x^2}{(2x^2-1)(2x+1)}=$\\
$\displaystyle =\lim_{x\to\infty}\frac{1+\frac{1}{x}}
{\left(2-\frac{1}{x^2}\right)\left(2+\frac{1}{x}\right)}=\frac14$.

\medskip
\noindent{\bf 288.} $\displaystyle\lim_{x\to\infty}
\frac{(x+1)^{10}+(x+2)^{10}+\ldots+(x+100)^{10}}{x^{10}+10^{10}}=$\\
$\displaystyle =\lim_{x\to\infty}
\frac{\left(1+\frac1x\right)^{10}+\left(1+\frac2x\right)^{10}+
\ldots+\left(1+\frac{100}{x}\right)^{10}}{1+\frac{10^{10}}{x^{10}}}=100$.

\medskip
\noindent{\bf 290.} $\displaystyle\lim_{x\to\infty}
\frac{\sqrt{x^2+1}-\sqrt[3]{x^2+1}}{\sqrt[4]{x^4+1}-\sqrt[5]{x^4+1}}=
\lim_{x\to\infty}\frac{x\left(\sqrt{1+\frac{1}{x^2}}-
\sqrt[3]{\frac{1}{x}+\frac{1}{x^3}}\right)}
{x\left(\sqrt[4]{1+\frac{1}{x^4}}-\sqrt[5]{\frac{1}{x}+\frac{1}{x^5}}\right)}=1$.

\medskip
\noindent{\bf 292.} $\displaystyle\lim_{x\to\infty}
\frac{\sqrt[3]{x^4+3}-\sqrt[5]{x^3+4}}{\sqrt[3]{x^7+1}}=$\\
$\displaystyle =\lim_{x\to\infty}\frac{x^{7/3}
\left(\sqrt[3]{\frac{1}{x^3}+\frac{3}{x^7}}-
\sqrt[5]{x^{3-35/3}+4x^{-35/3}}\right)}
{x^{7/3}\cdot\sqrt[3]{1+\frac{1}{x^7}}}=\frac01=0$.

\medskip
\noindent{\bf 294.} $\displaystyle\lim_{x\to 0}
\frac{\sqrt{1+x}-1}{x^2}=\quad|t=\sqrt{1+x};\quad x=t^2-1;\quad t\to 1.|\quad=$\\
$\displaystyle =\lim_{t\to 1}\frac{t-1}{(t^2-1)^2}=
\lim_{t\to 1}\frac{1}{(t-1)(t+1)^2}=\infty$.

\medskip
\noindent{\bf 299.} $\displaystyle\lim_{x\to 0}
\frac{\sqrt[3]{1+x^2}-1}{x^2}=\quad|t=
\sqrt[3]{1+x^2};\quad x^2=t^3-1;\quad t\to 1.|\quad=$\\
$\displaystyle =\lim_{t\to 1}\frac{t-1}{t^3-1}=
\lim_{t\to 1}\frac{1}{t^2+t+1}=\frac13$.

\medskip
\noindent{\bf 303.} $\displaystyle\lim_{x\to 0}
\frac{\sqrt[3]{1+x^2}-\sqrt[4]{1-2x}}{x+x^2}=
\lim_{x\to 0}\frac{\sqrt[3]{1+x^2}-1}{x+x^2}-
\lim_{x\to 0}\frac{\sqrt[4]{1-2x}-1}{x+x^2}=$\\
$\displaystyle =\lim_{x\to 0}\frac{1+x^2-1}
{x(x+1)[\sqrt[3]{(1+x^2)^2}+\sqrt[3]{1+x^2}+1]}-$\\
$\displaystyle-\lim_{x\to 0}\frac{1-2x-1}
{x(x+1)[\sqrt[4]{(1-2x)^3}+\sqrt[4]{(1-2x)^2}+\sqrt[4]{1-2x}+1]}=
0+\frac24=\frac12$.

\medskip
\noindent{\bf 304.} $\displaystyle\lim_{x\to 1}
\frac{\sqrt[3]{7+x^3}-\sqrt{3+x^2}}{x-1}=
\lim_{x\to 1}\frac{\sqrt[3]{7+x^3}-2}{x-1}-
\lim_{x\to 1}\frac{\sqrt{3+x^2}-2}{x-1}=$\\
$\displaystyle=\lim_{x\to 1}\frac{7+x^3-8}
{(x-1)[\sqrt[3]{(7+x^3)^2}+\sqrt[3]{7+x^3}\cdot2+4]}-
\lim_{x\to 1}\frac{3+x^2-4}{(x-1)(\sqrt{3+x^2}+2)}=$\\
$\displaystyle=\lim_{x\to 1}\frac{x^2+x+1}
{\sqrt[3]{(7+x^3)^2}+\sqrt[3]{7+x^3}\cdot2+4}-
\lim_{x\to 1}\frac{x+1}{\sqrt{3+x^2}+2}=\frac{3}{4+4+4}-\frac{2}{2+2}=-\frac14$.

\medskip
\noindent {\bf Используем первый замечательный предел и его следствия:}
$$\lim_{x\to0}\frac{\sin x}{x}=1,\qquad \lim_{x\to0}\frac{\tg x}{x}=1,\qquad
\lim_{x\to0}\frac{\arcsin x}{x}=1,\qquad \lim_{x\to0}\frac{\arctg x}{x}=1.$$

\medskip
\noindent Вычислить пределы:

\medskip
\noindent{\bf 320.} $\displaystyle\lim_{x\to 0}
\frac{2x-\arcsin x}{2x+\arctg x}=\lim_{x\to 0}
\frac{x\left(2-\frac{\arcsin x}{x}\right)}
{x\left(2+\frac{\arctg x}{x}\right)}=
\frac{2-1}{2+1}=\frac13$.

\medskip
\noindent{\bf 337.} $\displaystyle\lim_{x\to\pi}
\frac{1-\sin\frac{x}{2}}{\cos\frac{x}{2}(\cos\frac{x}{4}-\sin\frac{x}{4})}=
\quad|t=\pi-x;\quad x=\pi-t;\quad t\to 0.|\quad=$\\
$\displaystyle=\lim_{t\to 0}\frac{1-\cos\frac{t}{2}}{\sin\frac{t}{2}
\left[\cos\left(\frac{\pi}{4}-\frac{t}{4}\right)-
\sin\left(\frac{\pi}{4}-\frac{t}{4}\right)\right]}=$\\
$\displaystyle=\lim_{t\to 0}\frac{1-\cos\frac{t}{2}}{\sin\frac{t}{2}
\left(\cos\frac{\pi}{4}\cos\frac{t}{4}+\sin\frac{\pi}{4}\sin\frac{t}{4}-
\sin\frac{\pi}{4}\cos\frac{t}{4}+\cos\frac{\pi}{4}\sin\frac{t}{4}\right)}=$\\
$\displaystyle=\lim_{t\to 0}\frac{2\sin^2\frac{t}{4}}{\sin\frac{t}{2}
\left(\sqrt2\sin\frac{t}{4}\right)}=
\lim_{t\to 0}\frac{2t^2\cdot 2\cdot 4}
{4^2\cdot t\cdot\sqrt2\cdot t}=\frac{\sqrt2}{2}$.

\medskip
\noindent{\bf 345.} $\displaystyle\lim_{x\to 0}
\frac{\sqrt2-\sqrt{1+\cos x}}{\sin^2x}=
\lim_{x\to 0}\frac{2-1-\cos x}{\sin^2x\cdot(\sqrt2+\sqrt{1+\cos x})}=$\\
$\displaystyle=\lim_{x\to 0}\frac{2\sin^2\frac{x}{2}}
{\sin^2x\cdot(\sqrt2+\sqrt{1+\cos x})}=
\lim_{x\to 0}\frac{2\cdot x^2}{2^2\cdot x^2\cdot
(\sqrt2+\sqrt{1+\cos x})}=\frac{\sqrt2}{8}$.

\medskip
\noindent{\bf 349.} $\displaystyle\lim_{x\to 0}
\frac{\sqrt[3]{1+\arctg3x}-\sqrt[3]{1-\arcsin3x}}
{\sqrt{1-\arcsin2x}-\sqrt{1+\arctg2x}}=
\lim_{x\to 0}\frac{1+\arctg3x-1+\arcsin3x}{1-\arcsin2x-1-\arctg2x}\times$\\
$\displaystyle\times\lim_{x\to 0}\frac{\sqrt{1-\arcsin2x}+\sqrt{1+\arctg3x}}
{\sqrt[3]{(1+\arctg3x)^2}+\sqrt[3]{(1+\arctg3x)(1-\arcsin3x)}+
\sqrt[3]{(1-\arcsin3x)^2}}=$\\
$\displaystyle =\lim_{x\to 0}\frac{3x\cdot(\frac{\arctg3x}{3x}+\frac{\arcsin3x}{3x})}
{-2x\cdot(\frac{\arcsin2x}{2x}+\frac{\arctg2x}{2x})}\cdot\frac23=-1$.

\medskip
\noindent{\bf 350.} $\displaystyle\lim_{x\to -1}
\frac{\sqrt{\pi}-\sqrt{\arccos x}}{\sqrt{x+1}}=
\lim_{x\to -1}\frac{\pi-\arccos x}{\sqrt{x+1}\cdot(\sqrt{\pi}+\sqrt{\arccos x})}$.

\noindent
По смыслу задачи $\pi-\arccos x$ принадлежит первой четверти. Таким образом,
выбирая знак плюс для синуса этой величины, имеем для нашего случая
$\sin(\pi-\arccos x)=\sin\arccos x=+\sqrt{1-x^2}.$
Поэтому $\pi-\arccos x=\arcsin\sqrt{1-x^2}$.
Преобразуя числитель в соответствии с этим равенством, вычисляем предел:

\noindent$\displaystyle\lim_{x\to -1}\frac{\arcsin \sqrt{1-x^2}}
{\sqrt{x+1}\cdot(\sqrt{\pi}+\sqrt{\arccos x})}=
\lim_{x\to -1}\frac{\sqrt{1-x^2}}
{\sqrt{x+1}\cdot(\sqrt{\pi}+\sqrt{\arccos x})}=$\\
$\displaystyle=\lim_{x\to -1}\frac{\sqrt{1-x}}
{\sqrt{\pi}+\sqrt{\arccos x}}=\frac{\sqrt2}{2\sqrt{\pi}}$.

\medskip
\noindent {\bf Используем второй замечательный предел и его следствия:}
$$\lim_{x\to0}(1+x)^{1/x}=e,\qquad \lim_{x\to\infty}\left(1+\frac1x\right)^x=e,\qquad\lim_{x\to0}\frac{\ln(1+x)}{x}=1,$$
$$\lim_{x\to0}\frac{a^x-1}{x}=\ln a\mbox{ при }a>0\mbox{ и }a\ne1,\qquad\lim_{x\to0}\frac{(1+x)^\alpha-1}{\alpha x}=1\mbox{ при }\alpha\ne0.$$

\medskip
\noindent Вычислить пределы:

\medskip
\noindent{\bf 362.} $\displaystyle\lim_{x\to \infty}
\left(\frac{x^2-2x+1}{x^2-4x+2}\right)^x=
\lim_{x\to \infty}\left[\left(1+\frac{2x-1}{x^2-4x+2}\right)^
{\frac{x^2-4x+2}{2x-1}}\right]^{\frac{(2x-1)x}{x^2-4x+2}}=$\\
$\displaystyle=e^{\lim\limits_{x\to \infty}\frac{2+\frac{1}{x}}
{1-\frac{4}{x}+\frac{2}{x^2}}}=e^2$.

\medskip
\noindent{\bf 364.} $\displaystyle\lim_{x\to 0}
(1+\tg^2\sqrt{x})^{\frac{1}{2x}}=
\lim_{x\to 0}[(1+\tg^2\sqrt{x})^{\frac{1}{\tg^2\sqrt x}}]^
{\frac{\tg^2\sqrt x}{2x}}=
e^{\lim\limits_{x\to 0}\frac{x}{2x}}=\sqrt e$.

\medskip
\noindent{\bf 373.} $\displaystyle\lim_{x\to 0}
\frac{e^x-e^{-x}}{\sin x}=\lim_{x\to 0}\frac{x}{\sin x}\cdot
\lim_{x\to 0}\frac{e^x-e^{-x}}{x}=
1\cdot\lim_{x\to 0}\frac{2}{e^x}\cdot\lim_{x\to 0}\frac{e^{2x}-1}{2x}=2$.

\medskip
\noindent{\bf 380.} $\displaystyle\lim_{x\to \pm\infty}
x(\sqrt{x^2+\sqrt{x^4+1}}-x\sqrt2)=
\lim_{x\to \pm\infty}\frac{x(x^2+\sqrt{x^4+1}-2x^2)}
{\sqrt{x^2+\sqrt{x^4+1}}+x\sqrt2}=$\\
$\displaystyle=\lim_{x\to \pm\infty}\frac{1}{\left(\sqrt{1+\sqrt{1+\frac{1}{x^4}}}+\sqrt2\right)}\cdot
\lim_{x\to \pm\infty}\frac{x^4+1-x^4}{\sqrt{x^4+1}+x^2}=
\frac{1}{2\sqrt2}\cdot\frac{1}{\infty}=0$.

\medskip
\noindent{\bf 385.} $\displaystyle\lim_{x\to\infty}
\frac{x+\sin x}{x+\cos x}=\lim_{x\to\infty}\frac{x+\cos x}{x+\cos x}+
\lim_{x\to\infty}\frac{\sin x-\cos x}{x+\cos x}=1-0=1$.

\medskip
\noindent{\bf 391.} $\displaystyle\lim_{x\to \infty}
x^2(1-\cos\frac{1}{x})=\lim_{x\to\infty}x^2\cdot2\sin^2\frac{1}{2x}=
\lim_{x\to\infty}x^2\cdot\frac{2}{4x^2}=1/2$.

\medskip
\noindent{\bf 395.} $\displaystyle\lim_{x\to 0}
\frac{\arcsin x-\arctg x}{x^3}.$

\noindent
По смыслу задачи $\arcsin x$, $\arctg x$ и $\arcsin x-\arctg x$
близки к нулю и находятся в правой полуплоскости. Поэтому:

\noindent
$\displaystyle\sin(\arcsin x-\arctg x)=
x\cdot\cos\arctg x-\cos\arcsin x\cdot\sin\arctg x=$\\
$\displaystyle =\frac{x}{\sqrt{1+x^2}}-\frac{\sqrt{1-x^2}\cdot x}{\sqrt{1+x^2}}=
\frac{x(1-\sqrt{1-x^2})}{\sqrt{1+x^2}}$.

\noindent
В соответствии с этим равенством преобразуем числитель и вычисляем предел:

\noindent
$\displaystyle\lim_{x\to 0}\frac{\arcsin x-\arctg x}{x^3}=
\lim_{x\to 0}\frac{\arcsin\frac{x(1-\sqrt{1-x^2})}{\sqrt{1+x^2}}}{x^3}=
\lim_{x\to 0}\frac{x(1-\sqrt{1-x^2})}{\sqrt{1+x^2}\cdot x^3}=$\\
$\displaystyle=\lim_{x\to 0}\frac{x(1-1+x^2)}{\sqrt{1+x^2}\cdot x^3(1+\sqrt{1-x^2})}=
\lim_{x\to 0}\frac{1}{\sqrt{1+x^2}(1+\sqrt{1-x^2})}=1/2$.


\noindent{\bf 399.} $\displaystyle\lim_{x\to 0}
\left(\frac{\sin x}{x}\right)^{\frac{\sin x}{x-\sin x}}=
\lim_{x\to 0}\left[\left(1+\frac{\sin x-x}{x}\right)^{\frac{x}{\sin x-x}}\right]^
{\frac{\sin x-x}{x}\cdot\frac{\sin x}{x-\sin x}}=$\\
$\displaystyle=e^{\lim\limits_{x\to 0}\frac{\sin x(\sin x-x)}{x(x-\sin x)}}=1/e$.

\bigskip
\noindent{\scriptsize \copyright Alidoro, 2016. palva@mail.ru }

\end{document}
