\documentclass[a5paper,10pt]{article}
\oddsidemargin=0pt
\hoffset=-1.5cm
\voffset=-1.5cm
\topmargin=-1.5cm
\textwidth=12.8cm
\textheight=18.6cm
\usepackage[utf8]{inputenc}
\usepackage[T2A]{fontenc}
\usepackage[russian]{babel}
\usepackage[T2A]{fontenc}
\usepackage{latexsym}
\usepackage{amssymb}
\usepackage{amsmath}
\usepackage{bm}
\usepackage{graphicx}

\begin{document}

\noindent {\it
\section* {Степенные и логарифмические функции}
Мерзляк А.Г. - Алгебра и начала математического анализа. 11 класс.
Углублённый уровень (2023)}

\subsection* {\S 1. Степень с произвольным действительным показателем. Показательная функция}

{\bf 1.2.} Вычислите значение выражения:

\medskip
\noindent
1) $\displaystyle 5^{(\sqrt3-1)^2}:\left(\frac15\right)^{2\sqrt3}.$

\medskip
\noindent
$\displaystyle 5^{(\sqrt3-1)^2}:\left(\frac15\right)^{2\sqrt3}=5^{(\sqrt3-1)^2}:\left(5^{-1}\right)^{2\sqrt3}=5^{(\sqrt3-1)^2+2\sqrt3}=5^4=625,$ это ответ.

\medskip
\noindent
{\bf 1.16.} Упростите выражение:

\medskip
\noindent
1) $\displaystyle \frac{(a^{2\sqrt6}-1)(a^{\sqrt6}+a^{2\sqrt6}+a^{3\sqrt6})}{a^{4\sqrt6}-a^{\sqrt6}}.$

\medskip
\noindent
$\displaystyle \frac{(a^{2\sqrt6}-1)(a^{\sqrt6}+a^{2\sqrt6}+a^{3\sqrt6})}{a^{4\sqrt6}-a^{\sqrt6}}
=\frac{(a^{2\sqrt6}-1)a^{\sqrt6}(1+a^{\sqrt6}+(a^{\sqrt6})^2)}{a^{\sqrt6}((a^{\sqrt6})^3-1)}=\\
=\frac{(a^{\sqrt6})^3-1}{(a^{\sqrt6})^3-1}=1,$ это ответ.

\medskip
\noindent
{\bf 1.47.} Исследуйте на четность функцию $\displaystyle y=\frac{2^x-3^x}{2^x+3^x}.$

\medskip
\noindent
$\displaystyle y(-x)=\frac{2^{-x}-3^{-x}}{2^{-x}+3^{-x}}=
\frac{\frac{1}{2^{-x}}-\frac{1}{3^{-x}}}{\frac{1}{2^{-x}}+\frac{1}{3^{-x}}}=
\frac{(3^x-2^x)2^x3^x}{2^x3^x(3^x+2^x)}=-\frac{2^x-3^x}{2^x+3^x}.$\\[3pt]
Ответ: функция нечетная.

\subsection* {\S 2. Показательные уравнения}

{\bf 2.1.}  Решите уравнение:

\medskip
\noindent
9) $0{,}25^{x^2-4}=2^{x^2+1}$

\medskip
\noindent
$\displaystyle 0{,}25^{x^2-4}=2^{x^2+1};\quad (2^{-2})^{x^2-4}=
2^{x^2+1};\quad 2^{-2(x^2-4)}=2^{x^2+1};\\
-2(x^2-4)=x^2+1;\quad 3x^2=7;\quad x^2 = 7/3;\quad x =
 \pm\sqrt\frac73,$ это ответ.

\subsection* {\S 3. Показательные неравенства}

{\bf 3.9.}  Решите неравенство:

\medskip
\noindent
4) $\displaystyle \left(\tg\frac{\pi}{3}\right)^{x-1}>9^{-0{,}5}$

\medskip
\noindent
$\displaystyle \left(\tg\frac{\pi}{3}\right)^{x-1}>9^{-0{,}5};\quad
(\sqrt3)^{x-1}>((\sqrt3)^4)^{-0{,}5};\quad x-1>-4\cdot0{,}5;\quad x>-2+1;\\[3pt]
x>-1,$ это ответ.

\subsection* {\S 4. Логарифм и ее свойства}

{\bf 4.}

\subsection* {\S 5. Логарифмическая функция и ее свойства}

{\bf 5.}

\subsection* {\S 6. Логарифмические уравнения}

{\bf 6.}

\subsection* {\S 7. Логарифмические неравенства}

{\bf 7.}

\subsection* {\S 28. Упражнения для повторения курсов математики, алгебры, алгебры и начал анализа}

{\bf 28.356.}  Решите неравенство:

\medskip
\noindent
3) $0{,}5^{x+3}-0{,}5^{x+2}+0{,}5^{x+1}<0{,}375;$

\medskip
\noindent
$0{,}5^{x+3}-0{,}5^{x+2}+0{,}5^{x+1}<0{,}375;\quad
0{,}5^{x}\cdot0{,}5^3-0{,}5^{x}\cdot0{,}5^2+0{,}5^{x}\cdot0{,}5^1<0{,}375;\\
0{,}5^{x}\cdot(0{,}125-0{,}25+0{,}5)<0{,}375;\quad0{,}5^{x}\cdot0{,}375<0{,}375;\quad
0{,}5^{x}<1;\quad x>0,$ это ответ.

\end{document}
