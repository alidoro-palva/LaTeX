\documentclass[a5paper,10pt]{article}
\oddsidemargin=0pt
\hoffset=-1.5cm
\voffset=-1.5cm
\topmargin=-1.5cm
\textwidth=12.8cm
\textheight=18.6cm
\usepackage[utf8]{inputenc}
\usepackage[T2A]{fontenc}
\usepackage[russian]{babel}
\usepackage{latexsym}
\usepackage{amssymb}
\usepackage{amsmath}
\usepackage{bm}
\usepackage{graphicx}

\begin{document}

{\it Мордкович А. Г. и др. -- Алгебра, 10 класс. Задачник (проф. уровень) (2009).}

\section*{\S 28. Преобразование суммы тригонометрических\\
функций в произведение}

\medskip
\noindent
Представить в виде произведения:

\medskip
\noindent
{\bf 28.1.} б) $\sin20^\circ-\sin40^\circ=2\sin(-10^\circ)\cos30^\circ=
-\sqrt3\sin10^\circ.$

\medskip
\noindent
{\bf 28.2.} г) $\displaystyle \cos75^\circ-\cos15^\circ=
-2\sin45^\circ\sin30^\circ=-\frac{\sqrt2}{2}.$

\medskip
\noindent
{\bf 28.3.} в) $\displaystyle \sin\frac{\pi}{6}+\sin\frac{\pi}{7}=
2\sin\frac{13\pi}{84}\cos\frac{\pi}{84}.$

\medskip
\noindent
{\bf 28.4.} б) $\displaystyle \cos\frac{11\pi}{12}+\cos\frac{3\pi}{4}=
2\cos\frac{5\pi}{6}\cos\frac{\pi}{12}.$

\medskip
\noindent
{\bf 28.5.} г) $\sin(\alpha-2\beta)-\sin(\alpha+2\beta)=
2\sin(-2\beta)\cos\alpha=-2\sin2\beta\cos\alpha.$

\medskip
\noindent
{\bf 28.6.} б) $\displaystyle \tg\frac{\pi}{5}-\tg\frac{\pi}{10}=
\frac{\sin(\frac{\pi}{4}-\frac{\pi}{10})}{\cos\frac{\pi}{4}\cos\frac{\pi}{10}}=
\frac{\sin\frac{3\pi}{20}}{\cos\frac{\pi}{4}\cos\frac{\pi}{10}}.$

\medskip
\noindent
{\bf 28.7.}
а) $\displaystyle \frac12-\cos t= \cos\frac{\pi}{3}-\cos t=
-2\sin\left(\frac{\pi}{6}+\frac{t}{2}\right)\sin\left(\frac{\pi}{6}-\frac{t}{2}\right)=\\
2\sin\left(\frac{t}{2}+\frac{\pi}{6}\right)\sin\left(\frac{t}{2}-\frac{\pi}{6}\right).$

\medskip
\noindent
г) $\displaystyle \cos t+\sin t=\sin\left(t+\frac{\pi}{2}\right)+\sin t=
2\sin\left(t+\frac{\pi}{4}\right)\sin\frac{\pi}{4}=\sqrt2\sin\left(t+\frac{\pi}{4}\right).$

\medskip
\noindent
{\bf 28.8.}
а) $\displaystyle \sin 5x+2\sin 6x+\sin 7x=2\sin 6x+\sin 6x\cos(-x)=2\sin 6x\cdot(1+\cos x)=\\[3pt]
=4\sin 6x \cos^2\frac{x}{2}.$

\medskip
\noindent
{\bf 28.9.}
б) $\cos 2t-\cos 4t-\cos 6t+\cos 8t=-2\sin 3t\sin(-t)+2\sin 7t\sin(-t)=\\
2\sin t(\sin 3t-\sin 7t)=2\sin t \cdot 2\sin(-2t) \cos 5t=-4\sin t \sin2t \cos 5t.$

\medskip
\noindent
Докажите тождество:

\medskip
\noindent
{\bf 28.10.}
а) $\displaystyle \frac{\sin2\alpha+\sin6\alpha}{\cos2\alpha+\cos6\alpha}=\tg4\alpha.\\[3pt]
\frac{\sin2\alpha+\sin6\alpha}{\cos2\alpha+\cos6\alpha}=
\frac{2\sin4\alpha\cos(-2\alpha)}{2\cos4\alpha\cos(-2\alpha)}=\tg4\alpha.$

\medskip
\noindent
{\bf 28.12} a) $\displaystyle \sin x+\sin y+\sin(x-y)=
4\sin\frac{x}{2}\cos\frac{y}{2}\cos\frac{x-y}{2}.$\\
Левая часть тождества равна
$\displaystyle 2\sin\frac{x+y}{2}\cos\frac{x-y}{2}+2\sin\frac{x-y}{2}\cos\frac{x-y}{2}=\\
=2\cos\frac{x-y}{2}\left(\sin\frac{x+y}{2}+\sin\frac{x-y}{2}\right)=
2\cos\frac{x-y}{2}\cdot2\sin\frac{x}{2}\cos\frac{y}{2}=\\
=4\sin\frac{x}{2}\cos\frac{y}{2}\cos\frac{x-y}{2}.$

\medskip
\noindent
{\bf 28.13} б) $\cos^2(\alpha-\beta)-\cos^2(\alpha+\beta)=\sin2\alpha\sin2\beta.$\\
Левая часть тождества равна $(\cos(\alpha-\beta)-\cos(\alpha+\beta))
(\cos(\alpha-\beta)+\cos(\alpha+\beta))=\\
=-2\sin\alpha\sin(-\beta)\cdot2\cos\alpha\cos\beta=
\sin2\alpha\sin2\beta.$

\medskip
\noindent
Вычислите:

\medskip
\noindent
{\bf 28.14.}
а) $\displaystyle \frac{\cos68^\circ-\cos22^\circ}{\sin68^\circ-\sin22^\circ}=
\frac{-2\sin45^\circ\sin23^\circ}{2\sin23^\circ\cos45^\circ}=-\tg45^\circ=-1.$

\medskip
\noindent
в) $\displaystyle \frac{\sin130^\circ+\sin110^\circ}{\cos130^\circ+\cos110^\circ}=
\frac{2\sin120^\circ\cos10^\circ}{2\cos120^\circ\cos10^\circ}=\tg120^\circ=
\tg(180^\circ-60^\circ)=\\[3pt]
=-\tg60^\circ=-\sqrt3.$

\medskip
\noindent
{\bf 28.15} а) $\displaystyle \frac{\sin\alpha+\sin3\alpha+\sin5\alpha+\sin7\alpha}
{\cos\alpha+\cos3\alpha+\cos5\alpha+\cos7\alpha},$ если $\ctg4\alpha=0{,}2.$\\
$=\displaystyle \frac{(\sin\alpha+\sin7\alpha)+(\sin3\alpha+\sin5\alpha)}
{(\cos\alpha+\cos7\alpha)+(\cos3\alpha+\cos5\alpha)}=
\frac{2\sin4\alpha\cos(-3\alpha)+2\sin4\alpha\cos(-\alpha)}
{2\cos4\alpha\cos(-3\alpha)+2\cos4\alpha\cos(-\alpha)}=\\
=\frac{2\sin4\alpha\cdot2(\cos3\alpha+\cos\alpha)}
{2\cos4\alpha\cdot2(\cos3\alpha+\cos\alpha)}=\frac{1}{\ctg4\alpha}=\frac{1}{0{,}2}=5.$

\medskip
\noindent
{\bf 28.16} а) $\displaystyle \sin^2 10^\circ+\sin^2 130^\circ+\sin^2 110^\circ=\\[3pt]
\frac{1-\cos20^\circ}{2}+\frac{1-\cos260^\circ}{2}+\frac{1-\cos220^\circ}{2}=
\frac12\left(3-\cos20^\circ+\sin10^\circ+\cos40^\circ\right)=\\
\frac12\left(3-2\sin30^\circ\sin10^\circ+\sin10^\circ\right)=
\frac12\left(3-\sin10^\circ+\sin10^\circ\right)=\frac32.$

\medskip
\noindent
Проверьте равенство:

\medskip
\noindent
{\bf 28.18.}
в) $\cos12^\circ-\cos48^\circ=\sin18^\circ.$\\
$\cos12^\circ-\cos48^\circ=-2\sin30^\circ\sin(-18^\circ)=-2\cdot(1/2)(-\sin18^\circ)=\sin18^\circ.$\\
Ответ: верно.

\medskip
\noindent
Докажите, что верно равенство:

\medskip
\noindent
{\bf 28.19.} a) $\sin20^\circ+\sin40^\circ-\cos10^\circ=0.$\\[3pt]
$\sin20^\circ+\sin40^\circ-\cos10^\circ=2\sin30^\circ\cos(-10^\circ)-\cos10^\circ=
\cos10^\circ-\cos10^\circ=0.$

\medskip
\noindent
{\bf 28.20.} б) $\cos115^\circ-\cos35^\circ+\cos65^\circ+\cos25^\circ=\sin5^\circ.$\\
Левая часть равенства равна $(\cos115^\circ+\cos65^\circ)-(\cos35^\circ-\cos25^\circ)=\\
=2\cos90^\circ\cos25^\circ+2\sin30^\circ\sin5^\circ=0+2\cdot(1/2)+\sin5^\circ=\sin5^\circ.$

\medskip
\noindent
{\bf 28.21.} а) $\sin47^\circ+\sin61^\circ-\sin11^\circ-\sin25^\circ=\cos7^\circ.$\\[3pt]
$\displaystyle \sin47^\circ+\sin61^\circ-\sin11^\circ-\sin25^\circ=
2\sin54^\circ\cos(-7^\circ)-2\sin18^\circ\cos(-7^\circ)=\\[3pt]
=\cos7^\circ\cdot2(\sin54^\circ-\sin18^\circ)=
\cos7^\circ\cdot2\cdot2\sin18^\circ\cos36^\circ=\\[3pt]
=\cos7^\circ\cdot\frac{2\cdot2\sin18^\circ\cos18^\circ\cos36^\circ}{\cos18^\circ}=
\cos7^\circ\cdot\frac{2\sin36^\circ\cos36^\circ}{\sin72^\circ}=\\[3pt]
=\cos7^\circ\cdot\frac{\sin72^\circ}{\sin72^\circ}=\cos7^\circ.$

\medskip
\noindent
б) $\tg55^\circ-\tg35^\circ=2\tg20^\circ.$\\[3pt]
Имеем $\displaystyle \tg20^\circ=\frac{\tg55^\circ-\tg35^\circ}{1+\tg55^\circ\tg35^\circ}=
\frac{\tg55^\circ-\tg35^\circ}{1+\ctg35^\circ\tg35^\circ}=
\frac{\tg55^\circ-\tg35^\circ}{1+1}.$\\[3pt]
Отсюда $\displaystyle \tg55^\circ-\tg35^\circ=2\tg20^\circ.$

\medskip
\noindent
{\bf 28.22.} Докажите, что если $\alpha+\beta+\gamma=\pi,$ то выполняется равенство:\\
a) $\tg\alpha+\tg\beta+\tg\gamma=\tg\alpha\tg\beta\tg\gamma.$\\[3pt]
Имеем $\displaystyle \frac{\tg\alpha+\tg\beta}{1-\tg\alpha\tg\beta}=\tg(\alpha+\beta)=-\tg(\pi-(\alpha+\beta))=-\tg\gamma.$ Отсюда\\
$\tg\alpha+\tg\beta=-\tg\gamma(1-\tg\alpha\tg\beta)=-\tg\gamma+\tg\alpha\tg\beta\tg\gamma.$ Теперь получаем\\
$\tg\alpha+\tg\beta+\tg\gamma=\tg\alpha\tg\beta\tg\gamma.$

\medskip
\noindent
б) $\displaystyle \sin\alpha+\sin\beta+\sin\gamma=
4\cos\frac{\alpha}{2}\cos\frac{\beta}{2}\cos\frac{\gamma}{2}.$\\[3pt]
$\displaystyle \sin\alpha+\sin\beta+\sin\gamma=
2\sin\frac{\alpha+\beta}{2}\cos\frac{\alpha-\beta}{2}+2\sin\frac{\gamma}{2}\cos\frac{\gamma}{2}=\\[3pt]
=2\cos\left(\frac{\pi}{2}-\frac{\alpha+\beta}{2}\right)\cos\frac{\alpha-
\beta}{2}+2\cos\left(\frac{\pi}{2}-\frac{\gamma}{2}\right)\cos\frac{\gamma}{2}=\\[3pt]
=2\cos\frac{\gamma}{2}\cos\frac{\alpha-\beta}{2}+2\cos\frac{\alpha+\beta}{2}\cos\frac{\gamma}{2}=
2\cos\frac{\gamma}{2}\left(\cos\frac{\alpha-\beta}{2}+\cos\frac{\alpha+\beta}{2}\right)=\\[3pt]
2\cos\frac{\gamma}{2}\cdot2\cos\frac{\alpha}{2}\cos\frac{\beta}{2}=
4\cos\frac{\alpha}{2}\cos\frac{\beta}{2}\cos\frac{\gamma}{2}.$ 

\medskip
\noindent
{\bf 28.23.}
а) Зная, что $\sin2x+\sin2y=a,\ \cos2x+\cos2y=b\ (a\ne0,\ b\ne0)$,
вычислите $\tg(x+y).$\\[3pt]
Имеем $a=2\sin(x+y)\cos(x-y),\ b=2\cos(x+y)\cos(x-y).$\\[3pt]
Теперь $\displaystyle \frac{a}{b}=\frac{\sin(x+y)}{\cos(x+y)}=\tg(a+b).$

\medskip
\noindent
{\bf 28.24.} Докажите:\\
б) Если $2\cos x=\cos(x+2y),$ то $\ctg(x+y)-2\tg x=\tg x+\ctg y.$\\ 
Опечатка в задачнике. Доказываем соотношение $\ctg(x+y)-2\tg x=\ctg y$ или
$\ctg(x+y)-\ctg y=2\tg x.$ Для этого вычисляем:\\[3pt]
$\displaystyle \ctg(x+y)-\ctg y=\frac{\cos(x+y)}{\sin(x+y)}-\frac{\cos y}{\sin y}=
\frac{\sin y\cos(x+y)-\cos y\sin(x+y)}{\sin(x+y)\sin y}=\\
=\frac{\sin(-x)\cdot2}{\cos x-\cos(x+2y)}=\frac{-2\sin x}{\cos x-2\cos x}=
2\frac{\sin x}{\cos x}=2\tg x.$

\medskip
\noindent
{\bf 28.25.} Докажите:\\
а) Если $\cos^2x+\cos^2y=m,$ то $\cos(x+y)\cos(x-y)=m-1.$\\[3pt]
$\displaystyle \cos(x+y)\cos(x-y)=\frac{\cos2y+\cos2x}{2}=
\frac{2\cos^2y-1+2\cos^2x-1}{2}=\\
=\cos^2x+\cos^2y-1=m-1.$

\medskip
\noindent
Решите уравнение:

\medskip
\noindent
{\bf 28.26.}
в) $\cos x=\cos5x.\\
\cos x-\cos5x=0;\ -2\sin3x\sin(-2x)=0.$\\
$\displaystyle \sin3x=0;\ 3x=k\pi;\ x=\frac{k\pi}{3}.\quad \sin2x=0;\ 2x=k\pi;\ x=\frac{k\pi}{2}.$\\
Ответ: $\displaystyle \frac{k\pi}{3},\ \frac{k\pi}{2}.$

\medskip
\noindent
{\bf 28.27.}
а) $\sin x+\sin 2x+\sin 3x=0.$\\
$\displaystyle \sin 2x+2\sin2x\cos(-x)=0;\ \sin 2x(1+2\cos x)=0.\\
\sin2x=0;\ 2x=k\pi;\ x=\frac{k\pi}{2}.\ 
1+2\cos x=0;\ \cos x=-\frac12;\ x=\pm\frac{2\pi}{3}+2k\pi.$\\
Ответ: $\displaystyle \frac{k\pi}{2},\ \pm\frac{2\pi}{3}+2k\pi.$

\medskip
\noindent
{\bf 28.28.} г) $\sin(7\pi+x)=\cos(9\pi+2x).$\\
$-\sin x=-\cos2x;\ -\sin x=-1+2\sin^2x.$ Обозначим $y=\sin x$.\\[3pt]
$\displaystyle 2y^2+y-1=0.\ y_{1,2}=\frac{-1\pm\sqrt{1+8}}{4}=\frac{-1\pm3}{4};\ y_1=\frac12,\ y_2=-1.$\\[3pt]
$\displaystyle \sin x=\frac12;\ x=\frac{\pi}{6}+2k\pi,\ x=\frac{5\pi}{6}+2k\pi.\ \sin x=-1;\ \frac{9\pi}{6}+2k\pi.$\\[3pt]
Ответ: $\displaystyle x=\frac{\pi}{6}+\frac{2n\pi}{3}.$

\medskip
\noindent
{\bf 28.29.} а) $1+\cos6x=2\sin^25x.$\\[3pt]
$1-2\sin^25x+\cos6x=0;\ \cos10x+\cos6x=0.\ 2\cos8x\cos2x=0.$\\
$\displaystyle \cos8x=0;\ 8x=\frac{\pi}{2}+k\pi;\ x=\frac{\pi}{16}+\frac{k\pi}{8};
\ \cos2x=0;\ 2x=\frac{\pi}{2}+k\pi;\ x=\frac{\pi}{4}+\frac{k\pi}{2}.$\\
Ответ: $\displaystyle \frac{\pi}{16}+\frac{k\pi}{8},\ \frac{\pi}{4}+\frac{k\pi}{2}.$

\medskip
\noindent
{\bf 28.30.} б) $2\sin^23x-1=\cos^24x-\sin^24x.$\\[3pt]
$-\cos6x=\cos8x;\ \cos8x+\cos6x=0;\ 2\cos7x\cos x=0.$\\[3pt]
$\displaystyle \cos7x=0; 7x=\frac{\pi}{2}+2k\pi;\ x=\frac{\pi}{14}+\frac{2k\pi}{7};
\ \cos x=0; x=\frac{\pi}{2}+2k\pi.$\\[3pt]
Ответ: $\displaystyle \frac{\pi}{14}+\frac{2k\pi}{7},\ \frac{\pi}{2}+2k\pi.$
Ответ задачника неполный.

\medskip
\noindent
{\bf 28.31.} г) $\displaystyle \ctg\frac{x}{2}+\ctg\frac{3x}{2}=0.$\\[3pt]
$\displaystyle \frac{\cos\frac{x}{2}}{\sin\frac{x}{2}}+
\frac{\cos\frac{3x}{2}}{\sin\frac{3x}{2}}=0;
\ \frac{\sin\frac{3x}{2}\cos\frac{x}{2}+\cos\frac{3x}{2}\sin\frac{x}{2}}
{\sin\frac{x}{2}\sin\frac{3x}{2}}=0;\ \frac{\sin2x}{\sin\frac{x}{2}\sin\frac{3x}{2}}=0.$\\[3pt]
Числитель должен равняться нулю, $\displaystyle \sin2x=0;
\ 2x=k\pi;\ x=\frac{k\pi}{2},$ при этом знаменатель не должен равняться нулю,
т. е. $\displaystyle \sin\frac{x}{2}\ne0;\ \frac{x}{2}\ne k\pi;\ x\ne 2k\pi,$
а также $\displaystyle \sin\frac{3x}{2}\ne0;\ \frac{3x}{2}\ne k\pi;\ x\ne \frac{2k\pi}{3}.$ Таким образом, ответ можно записать в виде двух групп решений.\\[3pt]
Ответ: $\displaystyle \frac{\pi}{2}+k\pi,\ \pi+2k\pi.$

\medskip
\noindent
{\bf 28.32.} б) $\sin5x+\sin x+2\sin^2x=1.$\\
$2\sin3x\cos2x=1-2\sin^2x;\ 2\sin3x\cos2x=\cos2x;\ \cos2x(2\sin3x-1)=0.$\\
$\displaystyle \cos2x=0;\ 2x=\frac{\pi}{2}+k\pi;\ x=\frac{\pi}{4}+\frac{k\pi}{2}.$\\
$\displaystyle 2\sin3x=1;\ \sin3x=\frac{1}{2};\ 3x=(-1)^k\frac{\pi}{6}+k\pi;
\ x=(-1)^k\frac{\pi}{18}+\frac{k\pi}{3}.$\\
Ответ: $\displaystyle \frac{\pi}{4}+\frac{k\pi}{2},
\ (-1)^k\frac{\pi}{18}+\frac{k\pi}{3}.$

\medskip
\noindent
{\bf 28.35.} б) При каких значениях $x$ числа $a,\ b,\ c$ образуют
арифметическую прогрессию, если\\
а) $a=\cos7x,\ b=\cos2x,\ c=\cos11x.$\\
Составляем уравнение: $\displaystyle b-a=c-b;\ \cos2x-\cos7x=\cos11x-\cos2x;\\
\ -2\sin\frac92x\sin\left(-\frac52x\right)=-2\sin\frac{13}{2}x\sin\frac92x;
\ \sin\frac92x\left(\sin\frac52+\sin\frac{13}{2}x\right)=0;\\
\sin\frac92x\cdot 2\sin\frac92x\cos(-2x)=0.\\
\sin\frac92x=0;\ \frac92x=k\pi;\ x=\frac{2k\pi}{9}.\\
\cos2x=0;\ 2x=\frac{\pi}{2}+k\pi;\ x=\frac{\pi}{4}+\frac{k\pi}{2}.$\\
Ответ: $\displaystyle \frac{2k\pi}{9},\ \frac{\pi}{4}+\frac{k\pi}{2}.$

\medskip
\noindent
{\bf 28.36.} Решите неравенство:\\
б) $\displaystyle \cos\left(2x+\frac{\pi}{3}\right)+\cos\left(2x-\frac{\pi}{3}\right)>-\frac{1}{2}.$\\
$\displaystyle 2\cos2x\cos\frac{\pi}{3}>-\frac{1}{2};\ \cos2x>-\frac{1}{2};\ -\frac{2\pi}{3}+2k\pi<2x<\frac{2\pi}{3}+2k\pi;\\[3pt]
-\frac{\pi}{3}+k\pi<x<\frac{\pi}{3}+k\pi.$

\medskip
\noindent
{\bf 28.38.} Постройте график уравнения:\\
а) $\sin2x=\sin2y.$\\
$\sin2x-\sin2y=0;\ 2\sin(x-y)\cos(x+y)=0.\\
\sin(x-y)=0;\ x-y=k\pi;\ y=x-k\pi.$ Это соотношение задает семейство прямых с угловым коэффициентом $1$ и пересекающих ось $Oy$ в точках $0,\ \pm\pi,\ \pm2\pi,\ \pm3\pi$ и т. д.\\
$\displaystyle \cos(x+y)=0;\ x+y=\frac{\pi}{2}+k\pi;\ y=-x+\frac{\pi}{2}+k\pi.$ Это соотношение задает семейство прямых с угловым коэффициентом $-1$ и пересекающих ось $Oy$ в точках\\
$\displaystyle \frac{\pi}{2},\ \frac{\pi}{2}\pm\pi,\ \frac{\pi}{2}\pm2\pi,\ \frac{\pi}{2}\pm3\pi$ и т. д.\\[3pt]
Эти два семейства прямых образуют равномерную прямоугольную сетку на плоскости.

\bigskip
\noindent{\scriptsize \copyright Alidoro, 2022. palva@mail.ru }

\end{document}
