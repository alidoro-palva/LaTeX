\documentclass[a5paper,10pt]{article}
\oddsidemargin=0pt
\hoffset=-1.5cm
\voffset=-1.5cm
\topmargin=-1.5cm
\textwidth=12.8cm
\textheight=18.6cm
\usepackage[utf8]{inputenc}
\usepackage[T2A]{fontenc}
\usepackage[russian]{babel}
\usepackage[T2A]{fontenc}
\usepackage{latexsym}
\usepackage{amssymb}
\usepackage{amsmath}
\usepackage{bm}
\usepackage{graphicx}

\begin{document}

\noindent {\it 
Кострикин А. И. (ред.) -- Сборник задач по алгебре МЦНМО (2009).}

\bigskip
\noindent
{\bf 56.1.} Доказать, что во всякой группе:\\
а) пересечение любого набора подгрупп является подгруппой;\\
б) объединение двух подгрупп является подгруппой тогда и только тогда, когда одна из подгрупп содержится в другой;\\
в) если подгруппа $C$ содержится в объединении подгрупп $A$ и $B$, то либо $C\subset A$ либо $C\subset B$. 

\medskip
\noindent
$\blacktriangleleft$\\
a) Пусть $G_i,\ (i\in I)$ -- подгруппы группы $G$. Элементы $g_1,g_2$ принадлежат их пересечению   $\displaystyle g_1,g_2\in \bigcap_{i\in I} G_i$. Тогда для любых $i\in I$ $g_1,g_2\in G_i$; $g_1g_2\in G_i$; $g_1^{-1}\in G_i$. Отсюда следует $\displaystyle g_1g_2\in \bigcap_{i\in I} G_i$; $\displaystyle g_1^{-1}\in \bigcap_{i\in I} G_i$, т. е. $\displaystyle \bigcap_{i\in I} G_i$ подгруппа группы $G$.\\
б) Предположим, что объединение двух подгрупп $G_1$ и $G_2$ является подгруппой группы $G$ и в то же время эти подгруппы не входят одна в другую. То есть имеются два таких элемента $g_1,\ g_2$, что $g_1\in G_1$, $g_1\notin G_2$, $g_2\notin G_1$, $g_2\in G_2$. Пусть $g_1g_2\in G_1$ тогда $g_1^{-1}g_1g_2\in G_1$ и $g_2\in G_1$. Но мы выбрали такой элемент $g_2$, который не входит в $G_1$. Противоречие доказывает, что $g_1g_2$ не может принадлежать $G_1$. Симметричным образом, предполагая, что $g_1g_2\in G_2$, имеем $g_1g_2g_2^{-1}\in G_2$ и $g_1\in G_2$, что опять таки противоречит выбору $g_1$. Все эти противоречия показывают, что если объединение двух подгрупп является подгруппой, то одна из этих подгрупп входит в другую. Обратное утверждение очевидно.\\
в) Имеем $C\subset A\cup B$, откуда $C=(C\cap A)\cup (C\cap B)$. $C$ -- подгруппа, поэтому по пункту a) множества $C\cap A$ и $C\cap B$ также подгруппы. По пункту б) одна из групп $C\cap A$ и $C\cap B$ входит в другую, поэтому подгруппа $C$ совпадает либо с $C\cap A$ либо с $C\cap B$. Последнее означает, что $C$ либо входит в $A$, либо входит в $B$.
$\blacktriangleright$

\bigskip
\noindent
{\bf 56.2.} Доказать, что конечная подполугруппа любой группы является подгруппой. Верно ли это утверждение, если подполугруппа бесконечна? 

\medskip
\noindent
$\blacktriangleleft$ Пусть $S$ -- подполугруппа группы $G$. Групповая операция, ограниченная на $S$, остается ассоциативной. Результат операции двух элементов $S$ принадлежит $S$ по определению подполугруппы. Возьмем произольный элемент $s$ подполугруппы. Все натуральные степени элемента $s$ принадлежат подполугруппе. В силу конечности последней, этих степеней конечное число, поэтому порядок элемента $s$ конечен. Таким образом все элементы циклической подгруппы, порожденной элментом $s$, могут быть представлены неотрицательными степенями $s$ и, следовательно, принадлежат подполугруппе. Среди элементов этой циклической подгруппы присутствует единица группы, а также элемент $s^{-1}$. Таким образом, мы доказали, что, во-первых, единица группы принадлежит подполугруппе, во-вторых, для произвольного элемента $s$ подполугруппы обратный ему элемент также принадлежит подполугруппе. Все аксиомы группы выполняются для подполугруппы $S$.\\[10pt]
Утверждение для бесконечных подполугрупп неверно. Например, подполугруппа натуральных чисел по сложению не является подгруппой группы целых чисел.
$\blacktriangleright$

\bigskip
\noindent
{\bf 63.10.} Доказать, что все обратимые элементы кольца с единицей образуют группу относительно умножения.

\medskip
\noindent
$\blacktriangleleft$ Пусть $M$ -- множество всех обратимых элементов кольца и $a,b\in M$. Тогда в силу $abb^{-1}a^{-1}=e=b^{-1}a^{-1}ab$ элемент $b^{-1}a^{-1}$ будет обратным для $ab$, т. е. элемент $ab$ обратим. В силу $aa^{-1}=e=a^{-1}a$ элемент $a^{-1}$ также обратим. Единица кольца, очевидно, обратима. Мы видим, что множество $M$ замкнуто относительн операции умножения, содержит единицу и вместе с каждым элементом содержит ему обратный. Следовательно, $M$ -- группа.
$\blacktriangleright$

\bigskip
\noindent
{\bf 63.15.} Пусть $R$ -- кольцо с единицей $x,y\in R$. Доказать, что:\\
а) если произведения $xy$ и $yx$ обратимы, то элементы $x$ и $y$ также обратимы;\\
б) если $R$ без делителей нуля и произведение $xy$ обратимо, то $x$ и $y$ и обратимы;\\
в) без дополнительных предположений о кольце $R$ из обратимости произведения $xy$ не следует обратимость элементов $x$ и $y$;\\
г) если обратим элемент $1+ab$ то обратим и элемент $1+ba$.

\medskip
\noindent
$\blacktriangleleft$ $\blacktriangleright$

\end{document}
