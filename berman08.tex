\documentclass[a5paper,10pt]{article}
\oddsidemargin=0pt
\hoffset=-1.5cm
\voffset=-1.5cm
\topmargin=-1.5cm
\textwidth=12.8cm
\textheight=18.6cm
\usepackage[utf8]{inputenc}
\usepackage[russian]{babel}
\usepackage[T2A]{fontenc}
\usepackage{fontspec}
\setmainfont{Times New Roman}
\usepackage{latexsym}
\usepackage{amssymb}
\usepackage{amsmath}
\usepackage{bm}
\usepackage{graphicx}

\begin{document}

\noindent {\it Берман. Сборник задач по курсу математического анализа.
Издание двадцатое. М., 1985.}

\bigskip
\section* {Глава VIII. Применения интеграла}

\medskip
\noindent{\bf 2459.} Вычислить площадь фигуры, ограниченной параболами
$y^2+8x=16$ и $y^2-24x=48$.

\medskip
\noindent $\blacktriangleleft$ Запишем уравнение парабол в каноническом виде:
$y^2=-8(x-2)$ и $y^2=24(x+2)$ из уравнений видно, что первая парабола
имеет вершину в точке (2,0) и ее ветви направлены влево, вторая парабола
имеет вершину в точке (-2,0) и ее ветви направлены вправо. Абциссу точек
пересечения парабол находим из уравнения $-8(x-2)=24(x+2);\quad -x+2=3x+6);\quad x=-1$.
Теперь можем записать площадь:\\
$\displaystyle S=2\int_{-2}^{-1}\sqrt{48+24x}\,dx+2\int_{-1}^{2}\sqrt{16-8x}\,dx=$\\
$\displaystyle =2\int_{-2}^{-1}\sqrt{48+24x}\,dx+2\int_{-1}^{2}\sqrt{16-8x}\,dx=$\\
$\displaystyle =\frac{2}{24}\int_{-2}^{-1}\sqrt{48+24x}\,d(48+24x)-
\frac{2}{8}\int_{-1}^{2}\sqrt{16-8x}\,d(16-8x)=$\\
$\displaystyle =\left.\frac{1}{12}\cdot\frac23(48+24x)\sqrt{48+24x}\right|_{-2}^{-1}-
\left.\frac{1}{4}\cdot\frac23(16-8x)\sqrt{16-8x}\right|_{-1}^{2}=$\\
$\displaystyle =\frac{1}{18}\cdot24\sqrt{24}+\frac{1}{6}\cdot24\sqrt{24}=
\frac29\cdot48\sqrt{6}=\frac{32}{3}\sqrt6$. $\blacktriangleright$

\medskip
\noindent{\bf 2464.} Найти площадь фигуры, ограниченной дугой гиперболы
и ее хордой, проведенной из фокуса перпендикулярно к действительной оси.

\medskip
\noindent $\blacktriangleleft$ Вычислим интеграл, который нам понадобится
в дальнейшем. Имеем:\\
$\displaystyle I=\int\sqrt{x^2-a^2}\,dx=
x\sqrt{x^2-a^2}-\int\frac{x^2dx}{\sqrt{x^2-a^2}}=$\\
$\displaystyle =x\sqrt{x^2-a^2}-\int\frac{(x^2-a^2)\,dx}{\sqrt{x^2-a^2}}-a^2\int\frac{dx}{\sqrt{x^2-a^2}}=$\\
$\displaystyle =x\sqrt{x^2-a^2}-I-a^2\int\frac{dx}{\sqrt{x^2-a^2}}$. Отсюда находим:\\
$\displaystyle I=\frac x2\sqrt{x^2-a^2}-\frac{a^2}{2}\int\frac{dx}{\sqrt{x^2-a^2}}=
\frac x2\sqrt{x^2-a^2}-\frac{a^2}{2}\ln|x+\sqrt{x^2-a^2}|$.\\
Теперь находим площадь фигуры.\\
$\displaystyle S=\frac {2b}{a}\int_a^{\sqrt{a^2+b^2}}\sqrt{x^2-a^2}\,dx=
\frac{b}{a}\left.\left(x\sqrt{x^2-a^2}-a^2\ln|x+\sqrt{x^2-a^2}|\right )\right|_a^{\sqrt{a^2+b^2}}=$\\
$\displaystyle =\frac{b}{a}\cdot(\sqrt{a^2+b^2}\cdot b-a^2\ln(\sqrt{a^2+b^2}+b)+a^2\ln a)=
\frac{b^2c}{a}-ab\ln\frac{c+b}{a}$. $\blacktriangleright$

\medskip
\noindent{\bf 2465.} Окружность $x^2+y^2=a^2$ разбивается гиперболой
$x^2-2y^2=a^2/4$ на три части. Определить площади этих частей.

\medskip
\noindent $\blacktriangleleft$ От окружности радиусом $a$ и площадью $\pi a^2$
гипербола отрезает две симметричные дольки. Сначала вычислим площадь одной
такой дольки. Координаты точек пересечения окружности и гиперболы
находим из системы:\\
$\displaystyle \begin{cases} x^2+y^2=a^2\\x^2-2y^2=\frac{a^2}{4}\end{cases};\quad
\begin{cases} x^2+y^2=a^2\\-3y^2=-\frac{3a^2}{4}\end{cases};\quad
\begin{cases} x=\pm\frac{\sqrt3}{2}a\\y=\pm\frac{a}{2}\end{cases}$.\\
Гипербола пересекает ось $Ox$ в точках $x=\pm a/2$, а окружность пересекает
эту ось в точках $x=\pm a$. Теперь мы можем написать площадь правой дольки:\\
$\displaystyle S_1=\sqrt2\int_{a/2}^{a\sqrt3/2}\sqrt{x^2-a^2/4}\,dx+
2\int_{a\sqrt3/2}^{a}\sqrt{a^2-x^2}\,dx$.\\
Теперь нам понадобится два неопределенных интеграла:\\
$\displaystyle\int\sqrt{x^2-a^2}\,dx=
\frac x2\sqrt{x^2-a^2}-\frac{a^2}{2}\ln|x+\sqrt{x^2-a^2}|$,\\
$\displaystyle\int\sqrt{a^2-x^2}\,dx=
\frac x2\sqrt{a^2-x^2}+\frac{a^2}{2}\arcsin\frac xa$,\\
вычисление которых можно посмотреть в задачах {\bf 2464} и {\bf 1984} соответственно.
Пользуясь ими, получаем площадь дольки:\\
$\displaystyle S_1=\sqrt2\left.\left(\frac x2\sqrt{x^2-a^2/4}-
\frac{a^2}{8}\ln|x+\sqrt{x^2-a^2/4}|\right)\right|_{a/2}^{a\sqrt3/2}+\\
+\left.\left(x\sqrt{a^2-x^2}+
a^2\arcsin\frac xa\right)\right|_{a\sqrt3/2}^{a}=
\sqrt2\left(\frac{a\sqrt3}{4}\cdot\frac{a}{\sqrt2}-
\frac{a^2}{8}\ln\left(\frac{a\sqrt3}{2}+\frac{a}{\sqrt2}\right)\right)-\\
-\sqrt2\left(-\frac{a^2}{8}\ln\frac{a}{2}\right)+
\left(\frac{a^2\pi}{2}\right)-\left(\frac {a\sqrt3}{2}\cdot\frac a2+
\frac{a^2\pi}{3}\right)=$\\
$\displaystyle =\frac{a^2\sqrt3}{4}-\frac{a^2\sqrt2}{8}\ln\frac{a(\sqrt2+\sqrt3)}{2}+
\frac{a^2\sqrt2}{8}\ln\frac a2+
\frac{a^2\pi}{6}-\frac{a^2\sqrt3}{4}=$\\
$\displaystyle =a^2\left[\frac{\pi}{6}-\frac{\sqrt2}{8}\ln{(\sqrt2+\sqrt3)}\right]$.\\
Площадь второй дольки $S_2$ такая же, а площадь средней части получаем вычитанием
площадей долек из площади круга:\\
$\displaystyle S_3=\pi a^2-S_1-S_2=
\pi a^2-2a^2\left[\frac{\pi}{6}-\frac{\sqrt2}{8}\ln{(\sqrt2+\sqrt3)}\right]=$\\
$\displaystyle =a^2\left[\frac{2\pi}{3}+\frac{\sqrt2}{4}\ln{(\sqrt2+\sqrt3)}\right]$.
$\blacktriangleright$

\medskip
\noindent{\bf 2469.} Найти площадь фигуры, ограниченной осью ординат и линией\\
$x=y^2(y-1)$.

\medskip
\noindent $\blacktriangleleft$ График функции $x=y^2(y-1)$ пересекаяет ось ординат
в точках $x=0$ и $x=1$ и уходит между этими точками в отрицательную область. Площадь
этой области равна\\
$\displaystyle S=-\int_0^1y^2(y-1)\,dy=-\left.\left(\frac{y^4}{4}-
\frac{y^3}{3}\right)\right|_0^1=-\frac14+\frac13=\frac{1}{12}$.
$\blacktriangleright$

\medskip
\noindent{\bf 2475.} Найти площадь фигуры, ограниченной замкнутой линией
$y^2=x^2-x^4$.

\medskip
\noindent $\blacktriangleleft$ Кривая представляет собой восьмерку, симметричную
относительно осей $Ox$ и $Oy$ и пересекающую ось $Ox$ в точках $-1$, $0$ и $1$.
Площадь четверти этой восьмерки расположенной в первой четверти равна интегралу\\
$\displaystyle S/4=\int_0^1\sqrt{x^2-x^4}\,dx=-\frac12\int_0^1\sqrt{1-x^2}\,d(1-x^2)=$\\
$\displaystyle =-\left.\frac12\cdot\frac23(1-x^2)\sqrt{1-x^2}\right|_0^1=\frac13.\quad
S=\frac{4}{3}$. $\blacktriangleright$

\medskip
\noindent{\bf 2480.} Вычислить площадь криволинейной трапеции, ограниченной линией\\
$y=e^{-x}(x^2+3x+1)+e^2$, осью $Ox$ и двумя прямыми,
параллельным оси $Oy$, проведенными через точки экстремума функции $y$.

\medskip
\noindent $\blacktriangleleft$ Найдем точки экстремума. Берем производную
функции $y$:\\
$y'=(e^{-x}(x^2+3x+1)+e^2)'=-e^{-x}(x^2+3x+1)+e^{-x}(2x+3)=$\\
$\displaystyle =e^{-x}(-x^2-x+2)$.\\
Производная обращается в нуль, когда квадратный трехчлен равен нулю:\\
$\displaystyle -x^2-x+2=0\qquad x=\frac{1\pm\sqrt{1+8}}{-2}=
\frac{-1\mp3}{2};\quad x_1=-2,\quad x_2=1$.\\
В полученных точках проверяем знак второй производной:\\
$y''=(e^{-x}(-x^2-x+2))'=-e^{-x}(-x^2-x+2)+e^{-x}(-2x-1)=
e^{-x}(x^2-x-3).\\
y''(-2)>0,\quad y''(1)<0$.
То есть, в точке $x=-2$ мы имеем минимум, а в точке $x=1$ -- максимум.
Учитывая, что $y(-2)=e^2((-2)^2-3\cdot 2+1)-e^2=0$, делаем вывод, что функция $y$
больше нуля на интересующем нас интервале $(-2,1)$, а искомая площадь
выражается интегралом\\
$\displaystyle S=\int_{-2}^1(e^{-x}(x^2+3x+1)+e^2)\,dx=
-\int_{-2}^1(x^2+3x+1)\,de^{-x}+3e^2=$\\
$\displaystyle =-\left.(e^{-x}(x^2+3x+1))\right|_{-2}^1+\int_{-2}^1e^{-x}(2x+3)\,dx+3e^2=$\\
$\displaystyle =-(e^{-1}\cdot5-e^2\cdot(-1))+3e^2-\int_{-2}^1(2x+3)\,de^{-x}=$\\
$\displaystyle =2e^2-\frac5e-\left.(e^{-x}(2x+3))\right|_{-2}^1+2\int_{-2}^1e^{-x}\,dx=
2e^2-\frac5e-\frac5e-e^2-2\left.e^{-x}\right|_{-2}^1=
e^2-\frac{10}{e}-\frac{2}{e}+2e^2=3e^2-\frac{12}{e}$.
$\blacktriangleright$

\medskip
\noindent{\bf 2527.} Найти периметр одного из криволинейных треугольников,
ограниченных осью абцисс и линиями $y=\ln\cos x$ и $y=\ln\sin x$.

\medskip
\noindent $\blacktriangleleft$
Периметр складывается из отрезка $[0,\pi/2]$, расположенного на оси
$Ox$ и двух симметричных кривых, одна из которых вычисляется интегралом\\
$\displaystyle\int_0^{\pi/4} \sqrt{1+(\ln\cos x)'^2}\,dx=
\int_0^{\pi/4} \sqrt{1+\frac{\sin^2 x}{\cos^2 x}}\,dx=
\int_0^{\pi/4} \frac{dx}{\cos x}=$\\
$\displaystyle =\int_0^{\pi/4} \frac{d(\sin x)}{\cos^2 x}=
-\int_0^{\pi/4} \frac{d(\sin x)}{\sin^2 x-1}=
-\left.\frac12\ln\left|\frac{\sin x-1}{\sin x+1}\right|\right|_0^{\pi/4}=$\\
$\displaystyle =\frac12\ln\frac{2+\sqrt2}{2-\sqrt 2}=\frac12\ln\frac{(2+\sqrt2)^2}{2}=
\ln\frac{2+\sqrt2}{\sqrt2}=\ln(\sqrt2+1)$.\\
Сложив длины отрезка и двух кривых, получаем ответ:
$\displaystyle\frac{\pi}{2}+2\ln(\sqrt2+1)$. $\blacktriangleright$

\medskip
\noindent{\bf 2530.} Найти длину линии $(y-\arcsin x)^2=1-x^2$.

\medskip
\noindent $\blacktriangleleft$
$\displaystyle (y-\arcsin x)^2=1-x^2;\quad y=\arcsin x\pm\sqrt{1-x^2}.
x=\sin t,\quad y=t\pm\cos t.\\
L=2\int_{-\pi/2}^{\pi/2}\sqrt{\cos^2t+1-2\sin t+\sin^2t}\,dt=
2\sqrt2\int_{-\pi/2}^{\pi/2}\sqrt{1-\sin t}\,dt=$\\
$\displaystyle =4\sqrt2\int_{-\pi/2}^{\pi/2}
\sqrt{\cos^2\frac t2-2\cos\frac t2\sin\frac t2+\sin^2\frac t2}\,d\frac t2=$\\
$\displaystyle =4\sqrt2\int_{-\pi/2}^{\pi/2}
\left(\cos\frac t2-\sin\frac t2 \right)\,d\frac t2=
4\sqrt2\left.\left(\sin\frac t2+\cos\frac t2 \right)\right|_{-\pi/2}^{\pi/2}=
4\sqrt2\cdot\sqrt2=8$. $\blacktriangleright$

\medskip
\noindent{\bf 2534.} Найти длину линии $x=a\cos^5t$, $y=a\sin^5t$.

\medskip
\noindent $\blacktriangleleft$
$\displaystyle L=4\int_0^{\pi/2}\sqrt{25a^2\cos^8t\sin^2t+25a^2\sin^8t\cos^2t}\,dt=$\\
$\displaystyle =20a\int_0^{\pi/2}\cos t\sin t\sqrt{\cos^6t+\sin^6t}\,dt=$\\
$\displaystyle =10a\int_0^{\pi/2}\sqrt{(1-\sin^2t)^3+\sin^6t}\,d(\sin^2 t)=$\\
$\displaystyle =10a\int_0^{\pi/2}\sqrt{1-3\sin^2t+3\sin^4t}\,d(\sin^2 t)=$\\
$\displaystyle =10a\int_0^1\sqrt{1-3u+3u^2}\,du=10\sqrt3a\int_0^1
\sqrt{\left(u-\frac12\right)^2+\frac{1}{12}}\,d\left(u-\frac12\right)=$\\
$\displaystyle =5\sqrt3a\left.\left(\left(u-\frac12\right)
\sqrt{\left(u-\frac12\right)^2+\frac{1}{12}}-
\frac{1}{12}\ln\left(u-\frac12+
\sqrt{\left(u-\frac12\right)^2+
\frac{1}{12}}\right)\right)\right|_0^1=$\\
$\displaystyle =5\sqrt3a\left(\frac12\cdot\sqrt{\frac13}+
\frac{1}{12}\ln\left(\frac12+\sqrt{\frac13}\right)+
\frac12\cdot\sqrt{\frac13}-
\frac{1}{12}\ln\left(-\frac12+\sqrt{\frac13}\right)\right)=$\\
$\displaystyle =5a\left(1+\frac{\sqrt3}{12}\ln(2+\sqrt3)^2\right)=
5a\left(1+\frac{1}{2\sqrt3}\ln(2+\sqrt3)\right)$.
$\blacktriangleright$

\medskip
\noindent{\bf 2559.} Криволинейная трапеция, ограниченная линией $y=xe^x$
и прямыми $x=1$ и $y=0$ вращается вокруг оси абцисс. Найти объем тела,
которое при этом получается.

\medskip
\noindent $\blacktriangleleft$ $\displaystyle V=\pi\int_0^1(xe^x)^2dx=
\frac{\pi}{2}\int_0^1x^2\,d(e^{2x})=
\frac{\pi}{2}\left(\left.x^2e^{2x}\right|_0^1-
\int_0^1x\,d(e^{2x})\right)=$\\
$\displaystyle =\frac{\pi}{2}\left(e^2-\left.xe^{2x}\right|_0^1
+\frac12\int_0^1e^{2x}\,d(2x)\right)=
\frac{\pi}{2}\left(e^2-e^2+\frac12\left.e^{2x}\right|_0^1\right)=
\frac{\pi(e^2-1)}{4}$. $\blacktriangleright$

\medskip
\noindent{\bf 2563.} Найти объем тела, полученного от вращения
криволинейной трапеции, ограниченной линией $y=\arcsin x$,
с основанием $[0,1]$ вокруг оси $Ox$.

\medskip
\noindent $\blacktriangleleft$ $\displaystyle V=\pi\int_0^1\arcsin^2x\,dx=
\qquad\left|x=\sin t.\right|\qquad
=\pi\int_0^{\pi/2}t^2\,d\sin t=$\\
$\displaystyle =\pi\left(\left.(t^2\sin t)\right|_0^{\pi/2}-2\int_0^{\pi/2}t\sin t\,dt\right)=
\pi\left(\frac{\pi^2}{4}+2\int_0^{\pi/2}t\,d\cos t\right)=$\\
$\displaystyle =\pi\left(\frac{\pi^2}{4}+2\left.(t\cos t)\right|_0^{\pi/2}-
2\int_0^{\pi/2}\cos t\,dt\right)=
\pi\left(\frac{\pi^2}{4}+0-
2\left.\sin t\right|_0^{\pi/2}\right)=$\\
$\displaystyle =\pi\left(\frac{\pi^2}{4}-2\right)$. $\blacktriangleright$

\medskip
\noindent{\bf 2568.} Одна арка циклоиды $x=a(t-\sin t)$, $y=a(1-\cos t)$
вращается вокруг своего основания. Вычислить объем тела,
ограниченного полученной поверхностью.

\medskip
\noindent $\blacktriangleleft$ $\displaystyle V=\pi\int_0^{2\pi}y^2\,dx=
\pi\int_0^{2\pi}a^2(1-\cos t)^2\,d[a(t-\sin t)]=$\\
$\displaystyle =\pi a^3\int_0^{2\pi}(1-\cos t)^3\,dt=
\pi a^3\int_0^{2\pi}(1-3\cos t+3\cos^2t-\cos^3t)\,dt=$.\\
По формуле $\cos3t=4\cos^3t-3\cos t$ имеем:
$\displaystyle -\cos^3t=-\frac14\cos3t-\frac34\cos t$.\\
По формуле $\cos2t=2\cos^2t-1$ имеем:
$\displaystyle 3\cos^2t=\frac32\cos2t+\frac32$.\\
$\displaystyle =\pi a^3\int_0^{2\pi}
\left(\frac52-\frac{15}{4}\cos t+\frac32\cos2t-\frac14\cos3t \right)\,dt=$\\
$\displaystyle =\pi a^3\left.\left(\frac52t-\frac{15}{4}\sin t+\frac34\sin2t-
\frac{1}{12}\sin3t \right)\right|_0^{2\pi}=5\pi^2a^3$. $\blacktriangleright$

\medskip
\noindent{\bf 2584.} Вычислить объем тела, ограниченного параболоидом
$\displaystyle 2z=\frac{x^2}{4}+\frac{y^2}{9}$ и конусом
$\displaystyle \frac{x^2}{4}+\frac{y^2}{9}=z^2$.

\medskip
\noindent $\blacktriangleleft$ Тело образует кольцеобразную область,
расположенную между плоскостями $z=0$ и $z=2$. Точки тела находятся
вне конуса и внутри параболоида. Сечение тела плоскостью $z=z_0$
представляет собой эллипс с полуосями $2\sqrt{2z_0}$ и $3\sqrt{2z_0}$ из которого
выброшена внутренность в виде эллипса с полуосями  $2z_0$ и $3z_0$.
$\displaystyle V=\int_0^2(\pi\cdot2\sqrt{2z}\cdot3\sqrt{2z}-\pi\cdot2z\cdot3z)\,dz=
6\pi\int_0^2(2z-z^2)\,dz=
6\pi\left.\left(z^2-\frac{z^3}{3}\right)\right|_0^2=$\\
$\displaystyle =6\pi\left(4-\frac{8}{3}\right)=8\pi$. $\blacktriangleright$

\medskip
\noindent{\bf 2591.} Круг переменного радиуса перемещается таким образом,
что одна из точек его окружности остается на оси абсцисс, центр
движется по окружности $x^2+y^2=r^2$, а плоскость этого круга
перпендикулярна к оси абсцисс. Найти объем тела, которое при этом получается.

\medskip
\noindent $\blacktriangleleft$ $\displaystyle V=2\pi\int_{-r}^{r}(r^2-x^2)\,dx=
2\pi\left.\left(r^2x-\frac{x^3}{3}\right)\right |_{-r}^{r}=
2\pi\left(2r^3-\frac{2r^3}{3}\right)=\frac83\pi r^3$.
$\blacktriangleright$

\medskip
\noindent{\bf 2597.} При вращении эллипса
$\displaystyle \frac{x^2}{a^2}+\frac{y^2}{b^2}=1$
вокруг большой оси получается поверхность, называемая удлиненным
эллиспоидом вращения, при вращении вокруг малой -- поверхность,
называемая укороченным эллипсоидом вращения. Найти площадь поверхности
удлиненного и укороченного эллипсоидов вращения.

\smallskip
\noindent $\blacktriangleleft$\\
1) Удлиненный эллипсоид вращения.\\
$\displaystyle y=\frac ba\sqrt{a^2-x^2},\quad
y'=-\frac ba\cdot\frac{x}{\sqrt{a^2-x^2}},\quad
y'^2=\frac{b^2x^2}{a^2(a^2-x^2)}.\\
S=2\pi\int_{-a}^ay\sqrt{1+y'^2}\,dx=
4\pi\frac{b}{a}\int_{0}^a
\sqrt{a^2-x^2}\cdot\sqrt{\frac{a^4+(b^2-a^2)x^2}
{a^2(a^2-x^2)}}\,dx=$\\
$\displaystyle =4\pi\frac{b}{a^2}\int_{0}^a\sqrt{a^4-(a^2-b^2)x^2}\,dx=
4\pi\frac{b\sqrt{a^2-b^2}}{a^2}\int_{0}^a
\sqrt{\frac{a^4}{a^2-b^2}-x^2}\,dx=$\\
$\displaystyle =2\pi\frac{b\sqrt{a^2-b^2}}{a^2}\left.\left(x\sqrt{\frac{a^4}{a^2-b^2}-x^2}+
\frac{a^4}{a^2-b^2}\arcsin\frac{x\sqrt{a^2-b^2}}{a^2} \right)\right|_{0}^a=$\\
$\displaystyle =2\pi\frac{b\sqrt{a^2-b^2}}{a^2}\left(\frac{a^2b}{\sqrt{a^2-b^2}}+
\frac{a^4}{a^2-b^2}\arcsin\frac{\sqrt{a^2-b^2}}{a}\right)=$\\
$\displaystyle =2\pi b^2+\frac{2\pi ab\cdot a}{\sqrt{a^2-b^2}}\arcsin\frac{\sqrt{a^2-b^2}}{a}=
2\pi b^2+\frac{2\pi ab}{\varepsilon}\arcsin\varepsilon$.\\
2) Укороченный эллипсоид вращения.\\
$\displaystyle x=\frac ab\sqrt{b^2-y^2},\quad
x'=-\frac ab\cdot\frac{y}{\sqrt{b^2-y^2}},\quad
x'^2=\frac{a^2y^2}{b^2(b^2-y^2)}.\\
S=2\pi\int_{-b}^bx\sqrt{1+x'^2}\,dy=
4\pi\frac{a}{b}\int_{0}^b
\sqrt{b^2-y^2}\cdot\sqrt{\frac{b^4+(a^2-b^2)y^2}
{b^2(b^2-y^2)}}\,dy=$\\
$\displaystyle =4\pi\frac{a}{b^2}\int_{0}^b\sqrt{b^4+(a^2-b^2)y^2}\,dy=
4\pi\frac{a\sqrt{a^2-b^2}}{b^2}\int_{0}^b
\sqrt{\frac{b^4}{a^2-b^2}+y^2}\,dy=$\\
$\displaystyle =2\pi\frac{a\sqrt{a^2-b^2}}{b^2}\left.\left(
y\sqrt{\frac{b^4}{a^2-b^2}+y^2}-
\frac{b^4}{a^2-b^2}\ln\left|y+\sqrt{\frac{b^4}{a^2-b^2}+
y^2}\right|\right)\right|_{0}^b=$\\
$\displaystyle =2\pi a^2-\frac{2\pi ab^2}{\sqrt{a^2-b^2}}
\left(\ln\left|b+{\frac{ab}{\sqrt{a^2-b^2}}}\right|-
\ln\left|\frac{b^2}{\sqrt{a^2-b^2}}\right|\right)=$\\
$\displaystyle =2\pi a^2+\frac{2\pi ab^2}{\sqrt{a^2-b^2}}
\cdot\ln\frac{b}{\sqrt{a^2-b^2}+a}=
2\pi a^2+\frac{\pi ab^2}{\sqrt{a^2-b^2}}
\cdot\ln\frac{b^2}{(a+\sqrt{a^2-b^2})^2}=$\\
$\displaystyle =2\pi a^2+\frac{\pi ab^2}{\sqrt{a^2-b^2}}
\cdot\ln\frac{b^2(a-\sqrt{a^2-b^2})}{(a+\sqrt{a^2-b^2})(a^2-a^2+b^2)}=$\\
$\displaystyle =2\pi a^2+\frac{\pi ab^2}{\sqrt{a^2-b^2}}
\cdot\ln\frac{1-\frac{\sqrt{a^2-b^2}}{a}}{1+\frac{\sqrt{a^2-b^2}}{a}}=
2\pi a^2+\frac{\pi b^2}{\varepsilon}
\cdot\ln\frac{1-\varepsilon}{1+\varepsilon}$.\\
$\blacktriangleright$

\medskip
\noindent{\bf 2603.} Найти площадь поверхности, образованной вращением
астроиды\\
$x=a\cos^3t,\ y=a\sin^3t$ вокруг оси абсцисс.

\smallskip
\noindent $\blacktriangleleft$
$\displaystyle x=a\cos^3t,\ y=a\sin^3t.\\
S=4\pi\int_0^{\pi/2} y\sqrt{x'^2+y'^2}\,dt=
4\pi a^2\int_0^{\pi/2} \sin^3t\sqrt{9\cos^4t\sin^2t+9\sin^4t\cos^2t}\,dt=$\\
$\displaystyle =12\pi a^2\int_0^{\pi/2} \sin^4t\cos t\,dt=
12\pi a^2\int_0^{\pi/2} \sin^4t\,d(\sin t)=
\left.\frac{12}{5}\pi a^2\sin^5t\right|_0^{\pi/2}=\frac{12}{5}\pi a^2$.
$\blacktriangleright$

\medskip
\noindent{\bf 2610.} Вычислить статический момент прямоугольника
с основанием $a$ и высотой $h$ относительно его основания.

\smallskip
\noindent $\blacktriangleleft$
$\displaystyle M=\int_0^ha y\,dy=\left.\frac a2y^2\right|_0^h=\frac{ah^2}{2}$.
$\blacktriangleright$

\medskip
\noindent{\bf 2618.} Найти координаты центра масс фигуры,
ограниченной осями координат и параболой $\sqrt x+\sqrt y=\sqrt a$.

\smallskip
\noindent $\blacktriangleleft$
$P$ -- масса фигуры, $M_y$ -- статический момент фигуры относительно оси
$Oy$, $C_x$ и $C_y$ -- координаты центра тяжести.\\
$\displaystyle y=x-2\sqrt{ax}+a.\quad P=\int_0^a (x-2\sqrt{ax}+a)\,dx=
\left.\left(\frac{x^2}{2}-\frac{4\sqrt{a}}{3}x^{3/2}+ax\right)\right|_0^a=
\frac{a^2}{6}.\\
M_y=\int_0^a x(x-2\sqrt{ax}+a)\,dx=\left.\left(\frac{x^3}{3}-
\frac{4\sqrt a}{5}x^{5/2}+\frac{ax^2}{2}\right)\right|_0^a=$\\
$\displaystyle =\left(\frac13-\frac45+\frac12\right)a^3=\frac{a^3}{30}.\quad
C_x=\frac{M_y}{P}=\frac{a}{5}$. Симметричным образом
$\displaystyle C_y=\frac{a}{5}$. $\blacktriangleright$

\medskip
\noindent{\bf 2625.} Найти координаты центра масс фигуры,
ограниченной замкнутой линией  $y^2=ax^3-x^4$.

\smallskip
\noindent $\blacktriangleleft$
$\displaystyle y=x\sqrt{ax-x^2}.\quad P=2\int_0^ax\sqrt{ax-x^2}\,dx=$\\
$\displaystyle \left| \sqrt{ax-x^2}=xt,\quad ax-x^2=x^2t^2,\quad
x=\frac{a}{t^2+1},\quad dx=-\frac{2at\,dt}{(t^2+1)^2}. \right|$\\
$\displaystyle =-2\int_{-\infty}^0\frac{a}{t^2+1}\cdot\frac{at}{t^2+1}\cdot\frac{2at\,dt}{(t^2+1)^2}=
4a^3\left(\int_{-\infty}^0\frac{dt}{(t^2+1)^4}-\int_{-\infty}^0\frac{dt}{(t^2+1)^3} \right ).\\
M_x=2\int_0^ax^2\sqrt{ax-x^2}\,dx=
-2\int_{-\infty}^0\frac{a^2}{(t^2+1)^2}\cdot\frac{at}{t^2+1}\cdot\frac{2at\,dt}{(t^2+1)^2}=$\\
$\displaystyle =4a^4\left(\int_{-\infty}^0\frac{dt}{(t^2+1)^5}-
\int_{-\infty}^0\frac{dt}{(t^2+1)^4}\right).\\
I_n=\int_{-\infty}^0\frac{dx}{(x^2+1)^{n}}=
\left.\frac{x}{(x^2+1)^{n}}\right|_{-\infty}^0+
n\int_{-\infty}^0\frac{x\cdot 2x\,dx}{(x^2+1)^{n+1}}=$\\
$\displaystyle =0+2n\int_{-\infty}^0\frac{dx}{(x^2+1)^{n}}-
2n\int_{-\infty}^0\frac{dx}{(x^2+1)^{n+1}}=2nI_n-2nI_{n+1}.\\
I_{n+1}=\frac{2n-1}{2n}I_n.\\
I_4=\frac56I_3.\quad I_5=\frac78I_4=\frac{35}{48}I_3.\quad
C_x=\frac{4a^4(I_5-I_4)}{4a^3(I_4-I_3)}=
a\frac{\frac{35}{48}-\frac56}{\frac56-1}=\frac{5}{48}\cdot\frac61a=\frac58a.\\
C_y=0.$
$\blacktriangleright$

\medskip
\noindent{\bf 2634.} Найти центр масс сектора круга радиуса $R$
с центральным углом, равным $2\alpha$.

\smallskip
\noindent $\blacktriangleleft$ Центр тяжести лежит на биссектрисе
угла сектора. Расположим начало координат в центре круга, а ось $Ox$
направим по биссектрисе. Масса сектора равна $\alpha R^2$.
Статический момент сектора относительно оси $Oy$ равен\\
$\displaystyle M_y=2\int_0^{R\cos\alpha}\tan\alpha x^2dx+
2\int_{R\cos\alpha}^Rx\sqrt{R^2-x^2}\,dx=$\\
$\displaystyle =2\frac{R^3\tan\alpha \cos^3\alpha}{3}-
\int_{R\cos\alpha}^R\sqrt{R^2-x^2}\,d(R^2-x^2)=$\\
$\displaystyle =\frac{2R^3\tan\alpha \cos^3\alpha}{3}-
\frac23\left.(R^2-x^2)\sqrt{R^2-x^2}\right|_{R\cos\alpha}^R=$\\
$\displaystyle =\frac{2R^3\tan\alpha \cos^3\alpha}{3}+
\frac{2R^3\sin^3\alpha}{3}=\frac{2R^3\sin\alpha}{3}.\quad
C_x=\frac{2R\sin\alpha}{3\alpha},\quad C_y=0$.
$\blacktriangleright$

\medskip
\noindent{\bf 2640.} На каком расстоянии от геометрического центра
лежит центр масс полушара радиуса $R$?

\smallskip
\noindent $\blacktriangleleft$ Масса полушара равна $\displaystyle \frac{2\pi R^3}{3}$.
Расположим начало координат в центре шара и направим ось $Ox$
по оси симметрии. Тогда статический момент полушара относительно
плоскости $Oyz$ будет равен\\
$\displaystyle M_{yz}=\pi\int_0^Rx(R^2-x^2)\,dx=
\pi\left(\frac{R^4}{2}-\frac{R^4}{4}\right)=\pi\frac{R^4}{4}.\quad
C_x=\frac38R$. $\blacktriangleright$

\medskip
\noindent{\bf 2650.} Найти момент инерции полукруга радиуса $R$
относителльно его диаметра.

\smallskip
\noindent $\blacktriangleleft$
$\displaystyle I=2\int_0^R x^2\sqrt{R^2-x^2}\,dx= \qquad\left|x=R\sin t,\quad
dx=R\cos t\,dt\right|\\
=2\int_0^{\pi/2} R^2\sin^2x\cdot R\cos t\cdot R\cos t\,dt=
\frac{R^4}{2}\int_0^{\pi/2}\sin^22t\,dt=$\\
$\displaystyle =\frac{R^4}{4}\int_0^{\pi/2}(1-\cos4t)\,dt=\frac{\pi R^4}{8}$.
$\blacktriangleright$

\medskip
\noindent{\bf 2656.} Эллипс с полуосями $a$ и $b$ вращается вокруг одной из
своих осей. Найти момент инерции получающегося тела (эллипсоид вращения)
относительно оси вращения.

\smallskip
\noindent $\blacktriangleleft$ Бесконечно тонкий слой на расстоянии $x$ от
оси вращения $Oy$ и с толщиной $dx$ имеет форму цилиндра с радиусом $|x|$ и
высотой $\displaystyle \frac ba\sqrt{a^2-x^2}$ имеет момент инерции
$\displaystyle x^2\cdot 2\pi |x|\frac ba\sqrt{a^2-x^2}\,dx$. Поэтому
момент инерции тела вращения относительно оси $Oy$ выражается интегралом\\
$\displaystyle I_y=\frac{4\pi b}{a}\int_{0}^ax^3\sqrt{a^2-x^2}\,dx=
\frac{2\pi b}{a}\int_{0}^ax^2\sqrt{a^2-x^2}\,d(x^2)=\ldots\\
t^2=a^2-x^2,\quad x^2=a^2-t^2,\quad d(x^2)=-2t\,dt.\\
\ldots-=\frac{2\pi b}{a}\int_{a}^0(a^2-t^2)t\cdot2t\,dt=
\frac{4\pi b}{a}\int_{0}^a(a^2t^2-t^4)\,dt=
\frac{4\pi b}{a}\left(\frac{a^2a^3}{3}-\frac{a^5}{5}\right)=
\frac{8\pi a^4b}{15}$.\\
Симметричным образом, если мы будем вращать эллипс вокруг другой оси, то
получим тело с моментом инерции
$\displaystyle \frac{8\pi ab^4}{15}$.
$\blacktriangleright$

\medskip
\noindent{\bf 2657.} Найти момент инерции параболоида вращения, радиус
основания которого $R$, высота $H$, относительно оси вращения.

\smallskip
\noindent $\blacktriangleleft$ Рассмотрим часть параболы
$\displaystyle y=H\left(1-\frac{x^2}{R^2}\right)$, расположенную в первом октанте.
При ее вращении вокруг оси $Oy$ получается параболоид с параметрами,
описанными в условии задачи. Цилиндр толщины $dx$ радиуса $x$ и высоты $y$
имеет момент инерции $\displaystyle x^2\cdot2\pi xH\left(1-\frac{x^2}{R^2}\right)\,dx$.
Интегрируя этот момент от нуля до $R$ мы получим момент параболоида.\\
$\displaystyle I_y=2\pi H\int_0^Rx^3\left(1-\frac{x^2}{R^2}\right)\,dx=
2\pi H\left.\left(\frac{x^4}{4}-\frac{x^6}{6R^2}\right)\right|_0^R=\pi HR^4/6$.
$\blacktriangleright$

\medskip
\noindent{\bf 2659.} Криволинейная трапеция, ограниченная линиями
$y=e^x,\ y=0,\ x=0$ и $x=1$, вращается:
1) вокруг оси $Ox$, 2) вокруг оси $Oy$. Вычислить момент инерции
получающегося тела относительно оси вращения.

\smallskip
\noindent $\blacktriangleleft$\\
1) $\displaystyle I_x=\int_0^1 x^2\cdot2\pi x\,dx+
\int_1^e x^2\cdot2\pi x(1-\ln x)\,dx=
2\pi\int_0^e x^3dx-2\pi\int_1^e x^3\ln x\,dx=$\\
$\displaystyle =\frac{\pi e^4}{2}-\frac{\pi}{2}\int_1^e\ln x\,d(x^4)=
\frac{\pi e^4}{2}-\left.\frac{\pi}{2}x^4\ln x\right|_1^e+
\frac{\pi}{2}\int_1^ex^3\,dx=\frac{\pi}{2}\int_1^ex^3\,dx=$\\
$\displaystyle =\left.\frac{\pi}{8}x^4\right|_1^e=\frac{\pi}{8}(e^4-1)$.

\noindent 2) $\displaystyle I_y=\int_0^1 x^2\cdot2\pi x\cdot e^xdx=
2\pi\int_0^1 x^3e^xdx=
2\pi\left.x^3e^x\right|_0^1-6\pi\int_0^1x^2e^xdx=$\\
$\displaystyle =2\pi e-6\pi\left.x^2e^x\right|_0^1+12\pi \int_0^1xe^xdx=
-4\pi e+12\pi\left.xe^x\right|_0^1-12\pi\int_0^1e^xdx=$\\
$\displaystyle =8\pi e-12\pi\left.e^x\right|_0^1=-4\pi e+12\pi=4\pi(3-e)$.\\
$\blacktriangleright$

\medskip
\noindent{\bf 2664.} Эллипс с осями $AA_1=2a$ и $BB_1=2b$
вращается вокруг прямой, параллельной оси $AA_1$ и отстоящей от нее
на расстояние $3b$. Найти объем тела, которое при этом получается.

\smallskip
\noindent $\blacktriangleleft$ Площадь эллипса равнa $\pi ab$.
Центр тяжести эллипса находится в точке пересечения осей и
отстоит от оси вращения на расстояние $3b$. Длина окружности,
которую описывает центр тяжести при вращении равна $6\pi b$.
Применяя вторую теорему Гульдина получаем объем тела вращения
$6\pi^2 ab^2$. $\blacktriangleright$

\medskip
\noindent{\bf 2666.} Фигура, образованная первыми арками циклоид
$$x=a(t-\sin t),\quad y=a(1-\cos t)$$
и
$$x=a(t-\sin t),\quad y=-a(1-\cos t),$$
вращается вокруг оси ординат. Найти объем и поверхность тела, которое при
этом получается.

\smallskip
\noindent $\blacktriangleleft$ Точка описывает первую арку циклоиды,
когда параметр $t$ изменяется от $0$ до $2\pi$. Основание арки имеет
длину $2\pi a$. Из соображений симметрии центр тяжести фигуры образованной
арками отстоит от оси ординат на расстояние $\pi a$, а длина окружности,
которую описывает центр при вращении равна $2\pi^2a$. Теперь нам надо
вычислить площадь фигуры. Она равна интегралу\\
$\displaystyle S=2\int_0^{2\pi}y\,dx=
2a^2\int_0^{2\pi}(1-\cos t)\,d(t-\sin t)=
2a^2\int_0^{2\pi}(1-\cos t)^2\,dt=$\\
$\displaystyle =2a^2\cdot2\pi-4a^2\cdot0+2a^2\int_0^{2\pi}\cos^2t\,dt=
4\pi a^2+a^2\int_0^{2\pi}(1+\cos2t)\,dt=$\\
$\displaystyle =4\pi a^2+2\pi a^2+a^2\int_0^{2\pi}\cos2t\,dt=6\pi a^2$.\\
Используя вторую теорему Гульдина мы можем написать объем:\\
$V=6\pi a^2\cdot 2\pi^2a=12\pi^3a^3$.\\

\smallskip
\noindent Вычислим площадь поверхности. Центр тяжести контура фигуры
тот же, что и сама фигура, поэтому и длина окружности, которую
он описывает при вращении будет той же -- $2\pi^2a$. Теперь
вычислим длину контура фигуры. Она равна интегралу\\
$\displaystyle L=4\int_0^{\pi}\sqrt{x'^2+y'^2}\,dt=
4a\int_0^{\pi}\sqrt{(t-\sin t)'^2+(1-\cos t)'^2}\,dt=$\\
$\displaystyle =4a\int_0^{\pi}\sqrt{1-2\cos t+\cos t^2+\sin^2t}\,dt=
8a\int_0^{\pi}\sqrt{\frac{1-\cos t}{2}}\,dt=$\\
$\displaystyle =16a\int_0^{\pi}\sin\frac t2\,d\frac t2=
-\left.16a\cos\frac t2\right|_0^{\pi}=16a$.\\
Теперь по первой теореме Гульдина получаем площадь поверхности\\
$S_{\mbox{пов.}}=16a\cdot 2\pi^2a=32\pi^2a^2$. $\blacktriangleright$

\medskip
\noindent{\bf 2676.} С какой силой материальная ломаная $y=|x|+1$
притягивает материальную точку массы $m$, находящуюся в начале
координат? (Линейная плотность равна $\gamma$.)

\smallskip
\noindent $\blacktriangleleft$
Поскольку лучи ломаной симметричны относительно оси $Oy$, составляющие
сил тяготения от этих лучей, направленные вдоль оси $Ox$ уравновешивают
друг друга и дают нулевую сумму, а составляющие, направленные вдоль оси
$Oy$ равны и одинаково направлены. Таким образом, искомая сила тяготения
в два раза больше чем составляющая силы тяготения от одного луча,
направленная вдоль оси $Oy$. Возьмем луч, расположенный в правой полуплоскости.
Рассмотрим бесконечно малый участок луча с координатой $x$ и соответствующий
длине $dx$ оси $Ox$. Его масса равна $\gamma\sqrt 2\,dx$, расстояние участка
от начала координат равно $\sqrt{x^2+(x+1)^2}=\sqrt{2x^2+2x+1}$, а косинус
угла между направлением силы тяготения и осью $Oy$ равен
$\displaystyle\frac{x+1}{\sqrt{2x^2+2x+1}}$. Теперь мы можем
записать искомую силу интегралом:\\
$\displaystyle F=2\int_0^{\infty}\frac{km\sqrt 2\gamma}{2x^2+2x+1}\cdot
\frac{x+1}{\sqrt{2x^2+2x+1}}\,dx=
2\sqrt 2km\gamma\int_0^{\infty}\frac{(x+1)\,dx}{(2x^2+2x+1)^{3/2}}=$\\
$\displaystyle =2\sqrt 2km\gamma\int_0^{\infty}\frac{(x+1/2)\,dx}{(2x^2+2x+1)^{3/2}}+
\sqrt 2km\gamma\int_0^{\infty}\frac{dx}{(2x^2+2x+1)^{3/2}}$.\\
Первый интеграл приводится к интегралу степенной функции:\\
$\displaystyle \frac{\sqrt 2km\gamma}{2}\int_0^{\infty}
\frac{d(2x^2+2x+1)}{(2x^2+2x+1)^{3/2}}=
-\left.\frac{\sqrt 2km\gamma}{\sqrt{2x^2+2x+1}}\right|_0^{\infty}=
\sqrt 2km\gamma$.\\
Ко второму интегралу применяем подстановку Абеля:\\
$\displaystyle t=\frac{2x+1}{\sqrt{2x^2+2x+1}},\quad 2x+1=t\sqrt{2x^2+2x+1},\quad
4x^2+4x+1=t^2(2x^2+2x+1),\\
2(2x^2+2x+1)-1=t^2(2x^2+2x+1),\quad
2x^2+2x+1=\frac{1}{2-t^2},\\
2\,dx=dt\sqrt{2x^2+2x+1}+t^2dx,\quad (2-t^2)\,dx=\sqrt{2x^2+2x+1}\,dt,\\
\frac{dx}{\sqrt{2x^2+2x+1}}=\frac{dt}{2-t^2}.\\
\sqrt 2km\gamma\int_0^{\infty}\frac{dx}{(2x^2+2x+1)^{3/2}}=
\sqrt 2km\gamma\int_1^{\sqrt2}dt=\sqrt 2km\gamma(\sqrt2-1).$\\
Окончательный ответ: $\sqrt 2km\gamma+\sqrt 2km\gamma(\sqrt2-1)=2km\gamma$.
$\blacktriangleright$

\medskip
\noindent{\bf 2682.} Вычислить работу, которую необходимо затратить,
для того чтобы выкачать воду, наполняющую цилиндрический резервуар
высотой $H=5\mbox{ м}$, имеющий в основании круг радиуса $R=3\mbox{ м}$.

\smallskip
\noindent $\blacktriangleleft$
Бесконечно тонкий горизонтальный слой воды толщины $dx$ имеет объем $\pi R^2\,dx$
Его вес в ньютонах $\pi R^2 1000g\,dx$. Если слой расположен на глубине $x$, то
работа в джоулях, требующаяся для подъема воды этого слоя до уровня верхней кромки
резервуара, равна $\pi R^2 1000gx\,dx$. Работа по выкачиванию всей воды равна
интегралу\\
$\displaystyle \int_0^H\pi R^2 1000gx\,dx=\pi R^2 500gH^2=
3{,}14\cdot 3^2\cdot500\cdot10\cdot5^2=
3{,}5325\cdot10^6\mbox{ Дж}$.
$\blacktriangleright$

\medskip
\noindent{\bf 2691.} Круглый цилиндр, радиус основания которого
равен $R$, а высота $H$, вращается вокруг своей оси с постоянной
угловой скоростью $\omega$. Плотность материала, из которого
сделан цилиндр, равна $\gamma$. Найти кинетическую энергию цилиндра.

\smallskip
\noindent $\blacktriangleleft$
Кинетическая энергия равна $I\omega^2/2$, а момент инерции равен
интегралу\\
$I=\displaystyle \int_0^Rx^2\cdot2\pi xH\gamma\,dx=2\pi H\gamma\int_0^Rx^3dx=
2\pi H\gamma\cdot\frac{R^4}{4}$.\\
Теперь мы можем вычислить энергию, которая равна
$\pi R^4H\omega^2\gamma/4$. $\blacktriangleright$

\bigskip
\noindent{\scriptsize \copyright Alidoro, 2016. palva@mail.ru }

\end{document}
