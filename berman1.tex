\documentclass[a5paper,10pt]{article}
\oddsidemargin=0pt
\hoffset=-1.5cm
\voffset=-1.5cm
\topmargin=-1.5cm
\textwidth=12.8cm
\textheight=18.6cm
\usepackage[utf8]{inputenc}
\usepackage[russian]{babel}
\usepackage[T2A]{fontenc}
\usepackage{latexsym}
\usepackage{amssymb}
\usepackage{amsmath}
\usepackage{bm}
\usepackage{graphicx}

\begin{document}

\noindent {\it Берман. Сборник задач по курсу математического анализа.
Издание двадцатое. М., 1985.}

\medskip
\noindent{\bf 1563.} Найти наибольшее значение радиуса кривизны линии
$\displaystyle \rho=a\sin^3\frac{\varphi}{3}$.

\smallskip
\noindent $\blacktriangleleft$
$\blacktriangleright$

\medskip
\noindent Найти координаты центра кривизны и уравнение эволюты для данных линий.

\medskip
\noindent{\bf 1568.} Парабола $n\mbox{-го}$ порядка $y=x^n$.

\smallskip
\noindent $\blacktriangleleft$
$\blacktriangleright$

\medskip
\noindent{\bf 1574.} Линия $x=a(1+\cos^2t)\sin t,\quad y=a\sin^2t\cos t$.

\smallskip
\noindent $\blacktriangleleft$
$\blacktriangleright$

\medskip
\noindent{\bf 1581.} Показать, что эволютой астроиды $x=a\cos^3t,\ y=a\sin^3t$
является астроида вдвое бОльших линейных размеров, повернутая
на $45^\circ$. Воспользовавшись этим, вычислить длину дуги данной астроиды.

\smallskip
\noindent $\blacktriangleleft$
$\blacktriangleright$

\medskip
\noindent{\bf 1582*.} Показать, что эволюта кардиоиды
$$x=2a\cos t-a\cos 2t,\ y=2a\sin t-a\sin 2t$$
есть также кардиоида, подобная данной. Воспользовавшись этим,
найти длину дуги всей кардиоиды.

\smallskip
\noindent $\blacktriangleleft$
$\blacktriangleright$

\medskip
\noindent{\bf 3695.} Дан однородный шар радиуса $R$ с плотностью $\gamma$.
Вычислить силу, с которой он притягивает материальную точку массы $m$, находящуюся
на расстоянии $a\ (a>R)$ от его центра. Убедиться, что сила взаимодействия такова,
как если бы вся масса шара была сосредоточена в его центре.

\smallskip\noindent
$\blacktriangleleft$ Разместим начало координат в центре шара $D$, а ось $Oz$ направим в точку $M$
массы $m$, которая будет иметь координаты $(0,0,a)$. Рассмотрим точку шара $P$ с координатами
$(x,y,z)$. Ее расстояние до точки $M$ равно $\sqrt{x^2+y^2+(a-z)^2}$. Угол $\alpha=\angle PMO$
имеет косинус $\displaystyle\cos\alpha=\frac{a-z}{\sqrt{x^2+y^2+(a-z)^2}}$.
Из соображений симметрии сила тяготения будет направлена вдоль оси $Oz$ (в отрицательном
направлении). Чтобы получить компоненту силы тяготения, направленную вдоль $Ox$ надо
умножить величину этой силы на $\cos\alpha$, где $\alpha$ угол между направлением силы
$\overrightarrow{MP}$ и направлением $\overrightarrow{MO}$. Теперь мы можем записать
силу в виде тройного интеграла, который будем преобразовывать для сферической системы
координат\\
$\displaystyle F=\iiint_D\frac{\gamma m k(a-z)}{[x^2+y^2+(a-z)^2]^{3/2}}\,dx\,dy\,dz=\\
=\gamma m k\int_0^R dr\int_0^{\pi} d\theta
\int_0^{2\pi}\frac{r^2\sin\theta(a-r\cos\theta)}{[r^2\sin^2\theta+(a-r\cos\theta)^2]^{3/2}}\,d\varphi=\\
=-2\pi\gamma m k\int_0^Rr^2 dr\int_0^{\pi}
\frac{a-r\cos\theta}{(r^2+a^2-2ar\cos\theta)^{3/2}}\,d(\cos\theta)=$
$\blacktriangleright$

\medskip
\noindent{\bf 3698.} Дано однородное тело, ограниченное двумя концентрическими
сферами (шаровой слой). Доказать, что сила притяжения этим слоем точки,
находящейся во внутренней полости тела, равна нулю.

\smallskip\noindent
$\blacktriangleleft$ $\blacktriangleright$

\medskip
\noindent Вычислить несобственные интегралы или установить их расходимость:

\medskip
\noindent{\bf 3707.} $\displaystyle \int_0^{+\infty}\int_0^{+\infty}(x+y)e^{x+y}dx\,dy$.

\medskip
\noindent{\bf 3711.} $\displaystyle \int_0^{+\infty}dx\int_0^{+\infty}xe^{-y}\frac{\sin y}{y^2}\,dy$.

\medskip
\noindent Выяснить, какие из несобственных интегралов, взятых по кругу 
радиуса $R$ с центром в начале координат, являются сходящимися:

\medskip
\noindent{\bf 3712.} $\displaystyle\iint_D\ln\sqrt{x^2+y^2}\,dx\,dy$.

\medskip
\noindent{\bf 3714.} $\displaystyle\iint_D\frac{\sin(x^2+y^2)}{\sqrt{(x^2+y^2})^3}\,dx\,dy$.

\medskip
\noindent Вычислить несобственные интегралы:

\medskip
\noindent{\bf 3717.} $\displaystyle\int_0^{+\infty}\int_0^{+\infty}\int_0^{+\infty}
\frac{dx\,dy\,dz}{\sqrt{(1+x+y+z)^7}}$.

\medskip
\noindent{\bf 3719*.} $\displaystyle\int_0^{+\infty}\int_0^{+\infty}\int_0^{+\infty}
e^{-x^2-y^2-z^2}dx\,dy\,dz$.

\medskip
\noindent{\bf 3726.} Вычислить объем тела, ограниченного плоскостью $z=0$ и частью поверхности $z=xe^{-(x^2+y^2)}$, лежащей над этой плоскостью.

\medskip
\noindent{\bf 3731.} Найти кривизну линии $\displaystyle y=\int_{\pi}^{2\pi}\frac{\sin\alpha x}{\alpha}\,d\alpha$
в точке с абсциссой $x=1$.

\medskip
\noindent Вычислить интегралы с помощью дифференцирования по параметру:

\medskip
\noindent{\bf 3740.} $\displaystyle \int_0^1\frac{\ln(1-a^2x^2)}{x^2\sqrt{1-x^2}}\,dx\ (a^2<1)$.

\medskip
\noindent{\bf 3744.} $\displaystyle \int_0^{\pi/2}\ln\left(\frac{1+a\sin x}{1-a\sin x}\right)\frac{dx}{\sin x}\ (a^2<1)$.

\medskip
\noindent{\bf 3747*.} $\displaystyle \int_0^{+\infty}e^{-ax}\frac{\sin bx-\sin cx}{x}\,dx\ (a>0)$.

\medskip
\noindent{\bf 3750.} Вычислив интеграл $\displaystyle\int_0^{\pi/2}\frac{\arctg(a\tg x)}{\tg x}\,dx$,
найти $\displaystyle\int_0^{\pi/2}\frac{x}{\tg x}\,dx$.

\medskip
\noindent{\bf 3754.} Пусть функция $f(x)$ непрерывна при $x\ge0$ и при $x\to+\infty\ f(x)$
стремится к конечному пределу $f(+\infty)$. Доказать при этих условиях, что если
$a>0$ и $b>0$, то $\displaystyle\int_0^{+\infty}\frac{f(ax)-f(bx)}{x}\,dx=[f(+\infty)-f(0)]\ln\frac ab$.

\medskip
\noindent Вычислить интегралы, пользуясь результатом задачи 3754:

\medskip
\noindent{\bf 3755.} $\displaystyle\int_0^{+\infty}\frac{\arctg ax-\arctg bx}{x}\,dx$.

\medskip
\noindent{\bf 3757*.} Пусть функция $f(x)$ непрерывна при $x\ge0$ и
$\displaystyle \int_A^{+\infty}\frac{f(x)}{x}$
сходится при любом $A>0$. Доказать при этих условиях, что если $a>0$ и $b>0$, то\\
$\displaystyle \int_0^{+\infty}\frac{f(ax)-f(bx)}{x}\,dx=f(0)\ln\frac ba.$ (Ср. с задачей 3754.)

\medskip
\noindent Вычислить интегралы, пользуясь результатом задачи 3757 $(a>0,\ b>0)$:

\medskip
\noindent{\bf 3760.} $\displaystyle \int_0^{+\infty}\frac{\sin ax\sin bx}{x}\,dx$.

\medskip
\noindent{\bf 3763*.} Функция Лапласа $\Phi(x)$ определяется так:
$\displaystyle\Phi(x)=\frac{2}{\sqrt{\pi}}\int_0^xe^{-t^2}dt$
(эта функция играет больую роль в теории вероятностей). Доказать соотношения:\\
1) $\displaystyle\int_0^x\Phi(az)\,dz=\frac{e^{-a^2x^2}-1}{a\sqrt{\pi}}+x\Phi(ax)$;\qquad
2) $\displaystyle\int_0^{+\infty}[1-\Phi(x)]\,dx=\frac{1}{\sqrt{\pi}}$.

\medskip
\noindent{\bf 3765*.} Функция $J_0(x)$, определяемая равенством
$$J_0(x)=\frac{2}{\pi}\int_0^{\pi/2}\cos(x\sin\theta)\,d\theta,$$
называется функцией Бесселя нулевого порядка. Доказать, что:\\
1) $\displaystyle\int_0^{+\infty}e^{-ax}J_0(x)\,dx=\frac{1}{\sqrt{1+a^2}}\quad (a>0)$;\\
2) $\displaystyle\int_0^{+\infty}\frac{\sin ax}{x}J_0(x)\,dx=
\begin{cases}\pi/2, & \mbox{если }a\ge1;\\
\arcsin a, &\mbox{если }|a|\le 1;\\
-\pi/2, & \mbox{если }a\le-1.
\end{cases}$

\medskip
\noindent{\bf 3767*.} Доказать, что функция $\displaystyle y=\int_{-1}^1(z^2-1)e^{xz}dz$
удовлетворяет дифференциальному уравнению $xy''+2ny'-xy=0$.

\medskip
\noindent{\bf 3769*.} Доказать, что функция Бесселя нулевого порядка\\
$\displaystyle J_0(x)=\frac{2}{\pi}\int_0^{\pi/2}\cos(x\sin\theta)\,d\theta$
удовлетворяет дифференциальному уравнению $J_0''+\frac{J_0'(x)}{x}+J_0(x)=0$.

\medskip
\noindent Найти функции по данным полным дифференциалам.

\medskip
\noindent {\bf 3848.} $\displaystyle du=\frac{x}{y\sqrt{x^2+y^2}}\,dx-
\left(\frac{x^2+\sqrt{x^2+y^2}}{y^2\sqrt{x^2+y^2}}\right)\,dy$.

\medskip
\noindent {\bf 3851.} $\displaystyle du=\frac{2x(1-e^y)}{(1+x^2)^2}\,dx+
\left(\frac{e^y}{1+x^2}+1\right)\,dy$.

\medskip
\noindent {\bf 3854.} Подобрать постоянные $a$ и $b$ так. чтобы выражение\\
$\displaystyle \frac{y^2+2xy+ax^2)\,dx-(x^2+2xy+by^2)\,dy}{(x^2+y^2)^2}$ было полным
дифференциалом; найти соответствующую функцию.

\medskip
\noindent Найти функции по данным полным дифференциалам.

\medskip
\noindent {\bf 3856.} $\displaystyle du=\frac{x\,dx+y\,dy+z\,dz}{\sqrt{x^2+y^2+z^2}}$.

\medskip
\noindent {\bf 3859.} $\displaystyle du=\frac{dx-3\,dy}{z}+\frac{3y-x+z^3}{z^2}$.

\medskip
\noindent Найти общие решения данных дифференциальных уравнений

\medskip
\noindent {\bf 3903.} $\displaystyle yy'=\frac{1-2x}{y};\quad y^2dy=(1-2x)\,dx;$
$\displaystyle \frac{y^3}{3}=x-x^2+C;$\\
$y=\sqrt[3]{3x-3x^2+C}.$

\medskip
\noindent {\bf 3913.} $y'\sin x=y\ln y;\quad y|_{x=\pi/2}=e.$

\medskip
\noindent {\bf 3919.}

\medskip
\noindent {\bf 3923.}

\medskip
\noindent {\bf 3928.}

\medskip
\noindent {\bf 3933.}

\medskip
\noindent {\bf 3939.}

\medskip
\noindent {\bf 3944.}

\medskip
\noindent {\bf 3948.}

\medskip
\noindent {\bf 3952.}

\medskip
\noindent {\bf 3957.}

\medskip
\noindent {\bf 3963.}

\medskip
\noindent {\bf 3972.}

\medskip
\noindent {\bf 3977.}

\medskip
\noindent {\bf 3984.}

\medskip
\noindent {\bf 3990.}

\medskip
\noindent {\bf 3997.}

\smallskip
\noindent $\blacktriangleleft$
$\blacktriangleright$

\medskip
Улучшить сходимость тригонометрических рядов, доведя коэффициенты
до указанного в скобках порядка $k$:

\medskip
\noindent{\bf 4396*.} $\displaystyle \sum_{n=1}^{\infty}\frac{n^2}{n^3+1}\sin nx\quad (k=4)$.

\smallskip
\noindent $\blacktriangleleft$
$\blacktriangleright$

\medskip
\noindent{\bf 4399*.} $\displaystyle \sum_{n=2}^{\infty}\frac{n \sin\frac{n\pi}{2}}{n^2-1}\cos nx\quad (k=5)$.

\smallskip
\noindent $\blacktriangleleft$
$\blacktriangleright$

\medskip
\noindent{\bf 4414.} Вычислить $\mbox{div}(a\boldsymbol{r})$, где $a$ -- постоянный скаляр.

\smallskip
\noindent $\blacktriangleleft$ $\mbox{div}(ar)=\frac{\partial}{\partial x}ax+
\frac{\partial}{\partial y}ay+\frac{\partial}{\partial z}az=3a.$
$\blacktriangleright$

\medskip
\noindent{\bf 4460.} Вычислить поток радиус-вектора через боковую
поверхность круглого конуса, основание которого находится
на плоскости $xOy$, а ось совпадает с осью $Oz$. (Высота
конуса $1$, радиус основания $2$.)\bigskip

\smallskip
\noindent $\blacktriangleleft$
$\blacktriangleright$

\bigskip
\noindent{\scriptsize \copyright Alidoro, 2014. palva@mail.ru }

\end{document}
