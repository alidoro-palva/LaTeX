\documentclass[a5paper,10pt]{article}
\oddsidemargin=0pt
\hoffset=-1.5cm
\voffset=-1.5cm
\topmargin=-1.5cm
\textwidth=12.8cm
\textheight=18.6cm
\usepackage[utf8]{inputenc}
\usepackage[russian]{babel}
\usepackage[T2A]{fontenc}
\usepackage{fontspec}
\setmainfont{Times New Roman}
\usepackage{latexsym}
\usepackage{amssymb}
\usepackage{amsmath}
\usepackage{bm}
\usepackage{graphicx}

\begin{document}

\noindent {\it Берман. Сборник задач по курсу математического анализа.
Издание двадцатое. М., 1985.}

\bigskip
\section* {Глава VII. Способы вычисления определенных интегралов. Несобственные интегралы}

\medskip
\noindent{\bf 2237.} $\displaystyle \int_0^1(e^x-1)^4e^xdx=
\qquad\left|e^x=t\right|\qquad=\int_1^e(t-1)^4dt=
\left.\frac{(t-1)^5}{5}\right|_1^e=\frac{(e-1)^5}{5}$.

\medskip
\noindent{\bf 2245.} $\displaystyle \int_{1/2}^{\sqrt3/2}\frac{x^3dx}
{\left(\frac58-x^4\right)\sqrt{\left(\frac58-x^4\right)}}=
-\frac14\int_{1/2}^{\sqrt3/2}\frac{d\left(\frac58-x^4\right)}
{\left(\frac58-x^4\right)^{3/2}}=$\\
$\displaystyle =\left.\frac12\left(\frac58-x^4\right)^{-1/2}\right|_{1/2}^{\sqrt3/2}=
\frac12\left(\frac58-\frac{9}{16}\right)^{-1/2}-
\frac12\left(\frac58-\frac{1}{16}\right)^{-1/2}=
\frac{1}{2}(4-4/3)=4/3$.

\medskip
\noindent{\bf 2252.} $\displaystyle \int_{0}^{\pi/2}\cos^5x\sin2x\,dx=
-2\int_{0}^{\pi/2}\cos^6x\,d(\cos x)=
-\left.2\cdot\frac{\cos^7x}{7}\right|_{0}^{\pi/2}=\frac27$.

\medskip
\noindent{\bf 2260.} $\displaystyle \int_{0}^{\pi/2}x\cos x\,dx=
\left.x\sin x\right|_{0}^{\pi/2}-\int_{0}^{\pi/2}\sin x\,dx=
\frac{\pi}{2}+\left.\cos x\right|_{0}^{\pi/2}=\frac{\pi}{2}-1$.

\medskip
\noindent{\bf 2267.} $\displaystyle \int_{0}^{\pi/2}e^{2x}\cos x\,dx$.

\smallskip
\noindent $\blacktriangleleft$ $\displaystyle I=\int_{0}^{\pi/2}e^{2x}\cos x\,dx=
\left.e^{2x}\sin x\right|_{0}^{\pi/2}-2\int_{0}^{\pi/2}e^{2x}\sin x\,dx=$\\
$\displaystyle =e^{\pi}+2\int_{0}^{\pi/2}e^{2x}d(\cos x)=
e^{\pi}+\left.2e^{2x}\cos x\right|_{0}^{\pi/2}-4\int_{0}^{\pi/2}e^{2x}\cos x\,dx=$\\
$\displaystyle =e^{\pi}-2-4I;\quad I=\frac{e^{\pi}-2}{5}$. $\blacktriangleright$

\medskip
\noindent{\bf 2271.} Составить рекуррентную формулу и вычислить интеграл\\
$\displaystyle \int_{-1}^0x^ne^x\,dx$ ($n$ -- целое положительное число).

\smallskip
\noindent $\blacktriangleleft$ $\displaystyle \int_{-1}^0x^ne^x\,dx\\
I_n=\int_{-1}^0x^ne^x\,dx=x^ne^x\Big|_{-1}^0-n\int_{-1}^0x^{n-1}e^x\,dx=
-\frac{(-1)^n}{e}-nI_{n-1}.\\
I_0=1-\frac1e,\quad I_1=-1+\frac2e,\quad
I_2=2-\frac5e,\quad I_3=-2\cdot3+\frac{16}{e},\ldots\\
I_n=(-1)^nn!\left[1-\frac1e\left(\frac{1}{n!}+
\frac{1}{(n-1)!}+\ldots+\frac{1}{1!}+\frac{1}{0!}\right)\right]$.
$\blacktriangleright$

\medskip
\noindent{\bf 2281*.} $\displaystyle \int_0^{\pi}\sin^6\frac x2\,dx=
\int_0^{\pi}\left(\frac{1-\cos x}{2}\right)^3dx=$\\
$\displaystyle =\frac18\int_0^{\pi}(1-3\cos x+3\cos^2 x-\cos^3 x)\,dx=$\\
$\displaystyle =\frac{\pi}{8}-\frac38\sin x\Big|_0^{\pi}+
\frac38\int_0^{\pi}\frac{1+\cos2x}{2}\,dx-
\frac18\int_0^{\pi}(1-\sin^2x)d\sin x=$\\
$\displaystyle =\frac{\pi}{8}-0+\frac{3\pi}{16}+
\frac{3}{16}\sin 2x\Big|_0^{\pi}-\left(\frac{\sin x}{8}-
\frac{\sin^3x}{24} \right )\Big|_0^{\pi}=
\frac{5\pi}{16}$.

\medskip
\noindent{\bf 2288.} $\displaystyle \int_0^1\sqrt{(1-x^2)^3}\,dx=\qquad\left|x=\sin u,\quad dx=\cos x\,dx\right|\qquad\\
=\int_0^{\pi/2}\cos^4 u\,du=\int_0^{\pi/2}\frac{(1+\cos 2u)^2}{4}\,du=\\
=\int_0^{\pi/2}\frac{du}{4}+\int_0^{\pi/2}\frac{\cos 2u}{2}\,du+\int_0^{\pi/2}\frac{\cos^2 2u}{4}\,du=\\
=\frac{\pi}{8}+\frac{\sin 2u}{4}\Big|_0^{\pi/2}+\int_0^{\pi/2}\frac{1+\cos 4u}{8}\,du=
\frac{\pi}{8}+0+\frac{\pi}{16}+\int_0^{\pi/2}\frac{\cos 4u}{8}\,du=\\
=\frac{3\pi}{16}+\frac{\sin 4u}{32}\Big|_0^{\pi/2}=\frac{3\pi}{16}+0=\frac{3\pi}{16}.$

\medskip
\noindent{\bf 2295.} $\displaystyle \int_{\sqrt{8/3}}^{2\sqrt2}\frac{dx}{x\sqrt{(x^2-2)^5}}=\qquad
|x=\sqrt2\sec u,\quad dx=\sqrt2\sec u \tg u\,du|\\
=\int_{\pi/6}^{\pi/3}\frac{\sqrt2\sec u \tg u\,du}{\sqrt2\sec u\cdot 2^{5/2}\cdot \tg^5 u}=
\frac{\sqrt2}{8}\int_{\pi/6}^{\pi/3}\frac{du}{\tg^4u}=$\\
Отдельно вычислим неопределенный интеграл.\\
$\displaystyle \int\frac{du}{\tg^4u}=-\int\ctg^2u\cos^2u\,d\ctg u=
-\int\ctg^2u\cos^2u\,d\ctg u=\\
=-\int\ctg^2u\left(1-\frac{1}{1+\ctg^2u}\right)\,d\ctg u=
-\int\left(\ctg^2u-1+\frac{1}{1+\ctg^2u}\right)\,d\ctg u=\\
=-\frac{\ctg^3u}{3}+\ctg u+u.$\\
Теперь продолжаем вычисление определенного интеграла.\\
$\displaystyle =\frac{\sqrt2}{8}\left(-\frac{\ctg^3u}{3}+\ctg u+u\right)\Big|_{\pi/6}^{\pi/3}=
\frac{\sqrt2}{8}\left(-\frac{\sqrt3}{27}+\frac{\sqrt3}{3}+\frac{\pi}{3}+\sqrt3-\sqrt3-\frac{\pi}{6}\right)=\\
=\frac{\sqrt2}{8}\left(\frac{8\sqrt3}{27}+\frac{\pi}{6}\right)=\frac{\sqrt6}{27}+\frac{\pi\sqrt2}{48}.$

\medskip
\noindent{\bf 2298.} Вычислить среднее значение функций $f(x)=\sin x$ и
$f(x)=\sin^2x$ на отрезке $[0,\pi]$.

\smallskip
\noindent $\blacktriangleleft$ $\displaystyle \frac{\int_0^{\pi}\sin x\,dx}{\pi}=
\frac{-\cos x\big|_0^{\pi}}{\pi}=\frac {2}{\pi}.\\
\frac{\int_0^{\pi}\sin^2 x\,dx}{\pi}=\frac{\int_0^{\pi}(1-\cos 2x)\,dx}{2\pi}=
\frac{\int_0^{\pi}dx-\int_0^{\pi}\cos 2x\,d(2x)}{2\pi}=\frac{\pi-0}{2\pi}=\frac12.$
$\blacktriangleright$

\medskip
\noindent{\bf 2305.} $\displaystyle \int_0^2\frac{dx}{\sqrt{x+1}+\sqrt{(x+1)^3}}=\\
\left|t=\sqrt{x+1};\quad t^2=x+1;\quad x=t^2-1;\quad dx=2t\,dt.\right|\\
=\int_1^{\sqrt3}\frac{2t\,dt}{t+t^3}=\frac12\int_1^{\sqrt3}\frac{dt}{1+t^2}=
\frac12\arctg x\Big|_1^{\sqrt3}=\frac12\cdot\left(\frac{\pi}{3}-\frac{\pi}{4}\right)=
\frac{\pi}{6}$.

\medskip
\noindent{\bf 2312.} $\displaystyle \int_0^{\pi/2}\frac{dx}{2\cos x+3}=\qquad
\left|t=\tan\frac x2;\quad\cos x=\frac{1-t^2}{1+t^2};\quad dx=\frac{2\,dt}{1+t^2}\right|\\
=\int_0^1\frac{2\,dt}{\left(2\frac{1-t^2}{1+t^2}+3\right)(1+t^2)}=
\int_0^1\frac{2\,dt}{2(1-t^2)+3(1+t^2)}=2\int_0^1\frac{dt}{5+t^2}=\\
\frac{2}{\sqrt5}\arctg\frac{x}{\sqrt5}\Big|_0^1=
\frac{2}{\sqrt5}\arctg\frac{1}{\sqrt5}$.

\medskip
\noindent{\bf 2319.} Решить уравнение
$\displaystyle \int_{\sqrt2}^x\frac{dx}{x\sqrt{x^2-1}}=\frac{\pi}{12}$.

\smallskip
\noindent $\blacktriangleleft$ Сначала вычислим интеграл: $\displaystyle
\int_{\sqrt2}^x\frac{dx}{x\sqrt{x^2-1}}=
\frac12\int_{\sqrt2}^x\frac{d(x^2)}{x^2\sqrt{x^2-1}}=\\
\left|\sqrt{x^2-1}=t;\quad x^2=t^2+1;\quad d(x^2)=2t\,dt.\right|\\
=\frac12\int_1^{\sqrt{x^2-1}}\frac{2t\,dt}{(t^2+1)\cdot t}=\arctg t\Big|_1^{\sqrt{x^2-1}}=
\arctg\sqrt{x^2-1}-\frac{\pi}{4}$.\\

\smallskip\noindent Теперь можем решить уравнение:

\smallskip\noindent
$\displaystyle\arctg\sqrt{x^2-1}-\frac{\pi}{4}=\frac{\pi}{12};\quad
\arctg\sqrt{x^2-1}=\frac{\pi}{3};\quad \sqrt{x^2-1}=\sqrt{3};\quad x^2=4;\\
x=\pm2.$\\
Подынтегральное выражение в уравнении не имеет смысла на интервале
$(-1,1)$ поэтому при $x=-2$ интеграл также не имеет смысла. Ответ: $2$.
$\blacktriangleright$

\bigskip\noindent Вычислить несобственные интегралы (или установить их расходимость).

\medskip
\noindent{\bf 2371.} $\displaystyle \int_0^{+\infty}\frac{\ln x}{x}\,dx=
\int_0^{+\infty}\ln x\,d(\ln x)=\frac12(\ln x)^2\Big|_0^{+\infty}$.\\
Интеграл расходится.

\medskip
\noindent{\bf 2381.} $\displaystyle \int_0^{+\infty}e^{-ax}\cos bx\,dx$.

\smallskip
\noindent $\blacktriangleleft$ $\displaystyle I=\int_0^{+\infty}e^{-ax}\cos bx\,dx=
\frac1b\int_0^{+\infty}e^{-ax}d(\sin bx)=\\
=\frac1b\cdot e^{-ax}\sin bx\Big|_0^{+\infty}+\frac ab\int_0^{+\infty}e^{-ax}\sin bx\,dx=
0-\frac{a}{b^2}\int_0^{+\infty}e^{-ax}d(\cos bx)=\\
=-\frac{a}{b^2}\cdot e^{-ax}\cos bx\Big|_0^{+\infty}-
\frac{a^2}{b^2}\int_0^{+\infty}e^{-ax}\cos bx\,dx=
\frac{a}{b^2}-\frac{a^2}{b^2}I.\\
I=\frac{a}{b^2}-\frac{a^2}{b^2}I;\quad \left(1+\frac{a^2}{b^2}\right)I=
\frac{a}{b^2};\quad I=\frac{ab^2}{b^2(a^2+b^2)}=\frac{a}{a^2+b^2}$.
Ответ: $\displaystyle\frac{a}{a^2+b^2}$. $\blacktriangleright$

\bigskip\noindent Исследовать сходимость интегралов

\bigskip\noindent {\bf 2388.} $\displaystyle \int_0^{+\infty}\frac{x^{13}}{(x^5+x^3+1)^3}\,dx$.

\smallskip
\noindent $\blacktriangleleft$ Подынтегральная функция не превосходит функции,
интеграл от которой сходится:
$\displaystyle\frac{x^{13}}{(x^5+x^3+1)^3}<\frac{x^{13}}{(x^5)^3}=\frac{1}{x^2}$.
Ответ: Сходится. $\blacktriangleright$

\medskip
\noindent{\bf 2393.} $\displaystyle \int_e^{+\infty}\frac{dx}{x(\ln x)^{3/2}}$.

\smallskip
\noindent $\blacktriangleleft$ $\displaystyle \int_e^{+\infty}\frac{dx}{x(\ln x)^{3/2}}=
\int_e^{+\infty}\frac{d(\ln x)}{(\ln x)^{3/2}}=-2\cdot\frac{1}{(\ln x)^{1/2}}\Big|_e^{+\infty}.$
Ответ: Сходится. $\blacktriangleright$

\bigskip\noindent Вычислить несобственные интегралы (или установить их расходимость).

\bigskip
\noindent{\bf 2395.} $\displaystyle \int_0^2\frac{dx}{ x^2-4x+3}$.

\smallskip
\noindent $\blacktriangleleft$ $\displaystyle \int_0^2\frac{dx}{ x^2-4x+3}=
\int_0^2\frac{d(x-2)}{ (x-2)^2-1}=
\lim_{a\to1-0}\frac12\ln\frac{x-3}{x-1}\Big|_0^a+\lim_{a\to1+0}\frac12\ln\frac{x-3}{1-x}\Big|_a^2$.
Оба интеграла расходятся. Ответ: Расходится. $\blacktriangleright$

\medskip
\noindent{\bf 2403.} $\displaystyle \int_3^5\frac{x^2dx}{\sqrt{(x-3)(5-x)}}=
\qquad\Big|t=x-4;\quad x=t+4;\quad dx=dt.\Big|$\\
$\displaystyle =\int_{-1}^1\frac{t^2+8t+16}{\sqrt{(1+t)(1-t)}}\,dt=
-\int_{-1}^1\frac{1-t^2}{\sqrt{1-t^2}}\,dt
+4\int_{-1}^1\frac{2t\,dt}{\sqrt{1-t^2}}+17\int_{-1}^1\frac{dt}{\sqrt{1-t^2}}=$\\
$\displaystyle =-\int_{-1}^1\sqrt{1-t^2}\,dt
-4\int_{-1}^1\frac{d(1-t^2)}{\sqrt{1-t^2}}+17\int_{-1}^1\frac{dt}{\sqrt{1-t^2}}=$\\
Первый интеграл -- полукруг -- равен $\pi/2.$\\
$\displaystyle =-\frac{\pi}{2}-8\sqrt{1-t^2}\Big|_{-1}^1+17\arcsin x\Big|_{-1}^1=
-\frac{\pi}{2}-8\cdot0+17\pi=\frac{33}{2}\pi.$

\medskip
\noindent{\bf 2410.} $\displaystyle \int_{-1}^0\frac{e^{1/x}}{x^3}\,dx=\qquad \Big| x=\frac 1t;\quad
dx=-\frac{dt}{t^2};\quad t=\frac 1x.\Big|\qquad=\int_{-\infty}^{-1}te^tdt=$\\
$\displaystyle =te^t\Big|_{-\infty}^{-1}-\int_{-\infty}^{-1}e^tdt=-\frac 1e-\frac 1e=-\frac 2e$.

\bigskip\noindent Исследовать сходимость интегралов.

\medskip
\noindent{\bf 2417.} $\displaystyle \int_0^{\pi/2}\frac{\ln\sin x}{\sqrt x}\,dx$.

\smallskip
\noindent $\blacktriangleleft$ Подынтегральная функция стремится к $-\infty$ при $x\to0+0$.
При этом в некоторой окрестности нуля $\sin x>x/2$ и $\displaystyle \ln x>\frac{1}{\sqrt[3]x}$.
Поэтому отрицательную подынтегральную функцию можно оценить снизу следующим
образом: $$\displaystyle \frac{\ln\sin x}{\sqrt x}>\frac{\ln(x/2)}{\sqrt x}>-\frac{1}{\sqrt[3]{ x/2}\cdot\sqrt x}=-\frac{\sqrt[3]2}{x^{5/6}}.$$
Интеграл от последней функции сходится. Поэтому исходный интеграл от функции, которая больше  (меньше по абсолютной величине), также сходится. $\blacktriangleright$

\medskip
\noindent{\bf 2422.} Можно ли найти такое $k$, чтобы интеграл
$\displaystyle \int_{0}^{+\infty}x^kdx$ сходился?

\smallskip
\noindent $\blacktriangleleft$ При $k\le-1$ данный интеграл расходится в нуле,
при $k\ge-1$ интеграл расходится в $+\infty$. Таким образом, параметра $k$, при
котором интеграл сходился бы, не существует. $\blacktriangleright$

\bigskip\noindent Вычислить несобственные интегралы.

\medskip
\noindent{\bf 2429.} $\displaystyle \int_{0}^{+\infty}\frac{dx}{(a^2+x^2)^n}\quad$ ($n$ -- целое
положительное число).

\smallskip
\noindent $\blacktriangleleft$ Обозначим
$\displaystyle I_n=\int_{0}^{+\infty}\frac{dx}{(a^2+x^2)^n}$\\
$\displaystyle I_1=\int_{0}^{+\infty}\frac{dx}{(a^2+x^2)}=
\frac 1a\arctg\frac xa\Big|_{0}^{+\infty}=\frac{\pi}{2a}.$\\
Для $n>1$ имеем\\
$\displaystyle I_{n-1}=\int_{0}^{+\infty}\frac{dx}{(a^2+x^2)^{n-1}}=
\frac{dx}{(a^2+x^2)^{n-1}}\Big|_{0}^{+\infty}
+2(n-1)\int_{0}^{+\infty}\frac{x^2dx}{(a^2+x^2)^{n}}=$\\
$\displaystyle =0+2(n-1)\int_{0}^{+\infty}\frac{dx}{(a^2+x^2)^{n-1}}
-2a^2(n-1)\int_{0}^{+\infty}\frac{dx}{(a^2+x^2)^{n}}=$\\
$2(n-1)I_{n-1}-2a^2(n-1)I_n.$ Откуда $\displaystyle I_n=\frac{2n-3}{a^2(2n-2)}I_{n-1}.$\\
Ответ: $\displaystyle I_n=\frac{1\cdot3\cdot5\cdot\ldots\cdot(2n-3)}
{2\cdot4\cdot6\cdot\ldots\cdot(2n-2)}\cdot\frac{\pi}{2a^{2n-1}}.$ $\blacktriangleright$

\medskip
\noindent{\bf 2432.} $\displaystyle \int_{0}^{1}(\ln x)^ndx\quad$ ($n$ -- целое
положительное число).

\smallskip
\noindent $\blacktriangleleft$ Обозначим
$\displaystyle I_n=\int_{0}^{1}(\ln x)^ndx.\quad$
$\displaystyle I_1=\int_{0}^{1}\ln x\,dx=x\ln x\Big|_{0}^{1}-\int_{0}^{1}dx=0-1=-1.$\\
$\displaystyle I_n=\int_{0}^{1}(\ln x)^ndx=x(\ln x)\Big|_{0}^{1}-n\int_{0}^{1}(\ln x)^{n-1}dx=-nI_{n-1}.$\\
$I_2=1\cdot2,\quad I_3=-1\cdot2\cdot3,\quad I_4=1\cdot2\cdot3\cdot4,\quad\ldots,\quad I_n=(-1)^nn!.$ $\blacktriangleright$

\medskip
\noindent{\bf 2434*.} $\displaystyle \int_0^1\frac{(1-x)^n}{\sqrt x}\,dx\quad$ ($n$ -- целое
положительное число).

\smallskip
\noindent $\blacktriangleleft$ $\displaystyle \int_0^1\frac{(1-x)^n}{\sqrt x}\,dx=\qquad\Big|x=\sin^2t,\quad dx=2\sin t\cos t\,dt.\Big|$\\
$\displaystyle \qquad=\int_0^{\pi/2}\frac{(1-\sin^2t)^n\cdot2\sin t\cos t\,dt}{\sin t}
=2\int_0^{\pi/2}(\cos t)^{2n+1}dt=2I_{2n+1}.$\\
$\displaystyle I_{2n+1}=\int_0^{\pi/2}(\cos t)^{2n+1}dt=\int_0^{\pi/2}(\cos t)^{2n}d\sin t=$\\
$\displaystyle =(\cos t)^{2n}\sin t\Big|_0^{\pi/2}+2n\int_0^{\pi/2}(\cos t)^{2n-1}\sin^2t\,dt=$\\
$\displaystyle =0+2n\int_0^{\pi/2}(\cos t)^{2n-1}(1-\cos^2t)\,dt=2n(I_{2n-1}-I_{2n+1}).$\\
$\displaystyle (2n+1)I_{2n+1}=2nI_{2n-1};\quad I_{2n+1}=\frac{2n}{2n+1}I_{2n-1}.$\\
$\displaystyle I_{2n+1}=\frac{2n\cdot(2n-2)\ldots2}{(2n+1)\cdot(2n-1)\ldots3}I_1=\frac{2n\cdot(2n-2)\ldots2}{(2n+1)\cdot(2n-1)\ldots3}\int_0^{\pi/2}\cos t\,dt=$\\
$\displaystyle =\frac{2n\cdot(2n-2)\ldots2}{(2n+1)\cdot(2n-1)\ldots3}.$\qquad
Ответ: $\displaystyle 2\frac{2n\cdot(2n-2)\ldots2}{(2n+1)\cdot(2n-1)\ldots3}$. $\blacktriangleright$

\medskip
\noindent{\bf 2437*.} Доказать, что $\displaystyle \int_{0}^{+\infty}
\frac{x\ln x}{(1+x^2)^2}\,dx=0$.

\smallskip
\noindent $\blacktriangleleft$ Сначала выведем следующее равенство:\\
$\displaystyle \int_{1}^{+\infty}\frac{x\ln x}{(1+x^2)^2}\,dx=\qquad\Big|x=\frac1t,\quad dx=-\frac{dt}{t^2}\Big|\qquad =\int_{1}^{0}\frac{-\ln t}{t}\cdot\frac{t^4}{(1+t^2)^2}\cdot\frac{-1}{t^2}\,dt=$\\
$\displaystyle =\int_{1}^{0}\frac{t\ln t}{(1+t^2)^2}\,dt=-\int_{0}^{1}\frac{x\ln x}{(1+x^2)^2}\,dx.$\quad
Теперь можно написать:\\
$\displaystyle \int_{0}^{+\infty}\frac{x\ln x}{(1+x^2)^2}\,dx=
\int_{0}^{1}\frac{x\ln x}{(1+x^2)^2}\,dx+\int_{1}^{+\infty}\frac{x\ln x}{(1+x^2)^2}\,dx=0.$ $\blacktriangleright$

\bigskip\noindent Вычислить интегралы, пользуясь формулами
$$\int_0^{+\infty}e^{-x^2}dx=\frac{\sqrt\pi}{2}\mbox{ (интеграл Пуассона),}$$
$$\int_0^{+\infty}\frac{\sin x}{x}\,dx=\frac{\pi}{2}\mbox{ (интеграл Дирихле).}$$

\medskip
\noindent{\bf 2441*.} $\displaystyle \int_{0}^{+\infty}x^2e^{-x^2}dx=
-\frac12\int_{0}^{+\infty}x\,d(e^{-x^2})dx=$\\
$\displaystyle =-\frac12x\,e^{-x^2}\Big|_{0}^{+\infty}+\frac12\int_{0}^{+\infty}x^2e^{-x^2}dx=
0+\frac12\cdot\frac{\sqrt\pi}{2}=\frac{\sqrt\pi}{4}$.

\medskip
\noindent{\bf 2443.} $\displaystyle \int_{0}^{+\infty}\frac{\sin 2x}{x}\,dx=\int_{0}^{+\infty}\frac{\sin 2x}{2x}\,d(2x)=
\int_{0}^{+\infty}\frac{\sin t}{t}\,dt=\frac{\pi}{2}$.

\medskip
\noindent{\bf 2447.} $\displaystyle \int_{0}^{+\infty}\frac{\sin^3x}{x}\,dx$.

\smallskip
\noindent $\blacktriangleleft$ Имеем формулу: $\sin3x=3\sin x-4\sin^3x$. Откуда $\displaystyle \sin^3x=\frac34\sin x-\frac{\sin 3x}{4}$. Теперь можно написать\\
$\displaystyle \int_{0}^{+\infty}\frac{\sin^3x}{x}\,dx=\frac34\int_{0}^{+\infty}\frac{\sin x}{x}\,dx-
\frac14\int_{0}^{+\infty}\frac{\sin 3x}{x}\,dx=\frac34\cdot\frac{\pi}{2}-\frac14\cdot\frac{\pi}{2}=\frac{\pi}{4}$. $\blacktriangleright$

\bigskip\noindent Вычислить интегралы

\medskip
\noindent{\bf 2450.} $\displaystyle \int_{0}^{\pi/2}\ln \sin x\,dx$.

\smallskip
\noindent $\blacktriangleleft$ $\displaystyle I=\int_{0}^{\pi/2}\ln \sin x\,dx=
\int_{0}^{\pi/2}\ln\left(2\sin \frac x2\cos \frac x2\right)\,dx=$\\
$\displaystyle =\int_{0}^{\pi/2}\ln2\,dx+\int_{0}^{\pi/2}\ln\sin \frac x2\,dx+\int_{0}^{\pi/2}\ln\cos \frac x2\,dx=$\\
Первый интеграл вычисляем, во втором интеграле делаем замену $\displaystyle \frac x2=t$,
в третьем -- замену $\displaystyle \frac{\pi}{2}-\frac x2=t$.\\
$\displaystyle I=\frac{\pi}{2}\ln2+2\int_{0}^{\pi/4}\ln\sin t\,dt+2\int_{\pi/4}^{\pi/2}\ln\sin t\,dt=
\frac{\pi}{2}\ln2+2\int_{0}^{\pi/2}\ln\sin t\,dt=$\\
$\displaystyle =\frac{\pi}{2}\ln2+2I.$\quad Отсюда $\displaystyle I=-\frac{\pi}{2}\ln2$.\qquad Ответ:
$\displaystyle \int_{0}^{\pi/2}\ln \sin x\,dx=-\frac{\pi}{2}\ln2$. $\blacktriangleright$

\medskip
\noindent{\bf 2452*.} $\displaystyle \int_{0}^{\pi/2}x\cot x\,dx=
\int_{0}^{\pi/2}x\cdot\frac{d(\sin x)}{\sin x}=\int_{0}^{\pi/2}x\,d(\ln\sin x)=$\\
$\displaystyle =x\ln\sin x\Big|_{0}^{\pi/2}-\int_{0}^{\pi/2}\ln\sin x\,dx
\mbox { (Интеграл из задачи {\bf 2450.}) }=\frac{\pi}{2}\ln2$.

\medskip
\noindent{\bf 2454.} $\displaystyle \int_{0}^{1}\frac{\ln x\,dx}{\sqrt{1-x^2}}=
\qquad \Big|x=\sin t,\quad dx=\cos t\,dt.\Big|$\\
$\displaystyle =\int_{0}^{\pi/2}\frac{\ln\sin t\cdot\cos t\,dt}{\cos t}=
\int_{0}^{\pi/2}\ln\sin t\,dt\mbox { (Интеграл из задачи {\bf 2450.}) }=-\frac{\pi}{2}\ln2$.

\bigskip
\noindent{\scriptsize \copyright Alidoro, 2016. palva@mail.ru }

\end{document}
